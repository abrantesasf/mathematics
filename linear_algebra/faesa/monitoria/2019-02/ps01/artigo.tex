%%%%%%%%%%%%%%%%%%%%%%%%%%%%%%%%%%%%%%%%%%%%%%%%%%%%%%%%%%%%%%%%%%%%%%%%%%%%%%%%
% Modelo para artigo
%
% Por: Abrantes Araújo Silva Filho
%      abrantesasf@gmail.com


%%%%%%%%%%%%%%%%%%%%%%%%%%%%%%%%%%%%%%%%%%%%%%%%%%%%%%%%%%%%%%%%%%%%%%%%%%%%%%%%
%%% Classe do documento
\RequirePackage{ifpdf}
\ifpdf
  \documentclass[pdftex, brazil, 12pt, twoside]{article}
\else
  \documentclass[brazil, 12pt, twoside]{article}
\fi


%%%%%%%%%%%%%%%%%%%%%%%%%%%%%%%%%%%%%%%%%%%%%%%%%%%%%%%%%%%%%%%%%%%%%%%%%%%%%%%%
%%% Preâmbulo com todas as outras outras chamadas para todos os outros packages
%%% e o que mais for necessário
\input{utils/preambulo.tex}


%%%%%%%%%%%%%%%%%%%%%%%%%%%%%%%%%%%%%%%%%%%%%%%%%%%%%%%%%%%%%%%%%%%%%%%%%%%%%%%%
%%% Ajuste do layout, espaçamento de linhas e etc.:
\geometry{a4paper, left=2cm, right=2cm, top=2cm, bottom=2cm}
\onehalfspacing


%%%%%%%%%%%%%%%%%%%%%%%%%%%%%%%%%%%%%%%%%%%%%%%%%%%%%%%%%%%%%%%%%%%%%%%%%%%%%%%%
%%% Configurações para as propriedades do PDF:
\hypersetup{
      pdftitle={PS01: Previsão do tempo},
      pdfauthor={Abrantes Araújo Silva Filho},
      pdfsubject={Álgebra linear},
      pdfkeywords={cadeias de Markov},
      pdfinfo={
        CreationDate={}, % Ex.: D:AAAAMMDDHH24MISS
        ModDate={}       % Ex.: D:AAAAMMDDHH24MISS
      }}


%%%%%%%%%%%%%%%%%%%%%%%%%%%%%%%%%%%%%%%%%%%%%%%%%%%%%%%%%%%%%%%%%%%%%%%%%%%%%%%%
%%% Compilação condicional de capítulos
%\includeonly{}


%%%%%%%%%%%%%%%%%%%%%%%%%%%%%%%%%%%%%%%%%%%%%%%%%%%%%%%%%%%%%%%%%%%%%%%%%%%%%%%%
%%% Começa o documento
\begin{document}


%%%%%%%%%%%%%%%%%%%%%%%%%%%%%%%%%%%%%%%%%%%%%%%%%%%%%%%%%%%%%%%%%%%%%%%%%%%%%%%%
%%% Front matter
\title{\ingles{Problem Set 1}\\%
{\Large --- previsão do tempo com álgebra linear ---}}
\author{Abrantes Araújo Silva Filho}
\date{2019-08-30}
\maketitle

%\renewcommand{\abstractname}{novo título do abstract}
\abstract{Este \ingles{Problem Set} foi criado para familiarizar os alunos da
disciplina de álgebra linear com
o GNU Octave e apresentar uma aplicação prática da álgebra linear: a previsão
do tempo através de Cadeias de Markov.}

\tableofcontents


%%%%%%%%%%%%%%%%%%%%%%%%%%%%%%%%%%%%%%%%%%%%%%%%%%%%%%%%%%%%%%%%%%%%%%%%%%%%%%%%
%%% Main matter
\section{Introdução}
\label{sec:introd}

O \emph{Problem Set 1} (PS1) tem dois propósitos principais e bem
definidos:

\begin{enumerate}
\item Fazer com que você se
familiarize com o \href{https://www.gnu.org/software/octave/}{GNU Octave}, um
software que fornece um ambiente e uma linguagem interpretada de alto nível para
programação científica, com poderosa sintaxe orientada para matemática, cálculo
numérico, solução de problemas lineares e não lineares, extensa capacidade
gráfica e de visualização de dados~\citep{octave} e que, com raríssimas
exceções\footnote{De fato, o Octave foi criado para ser tão compatível com o
MATLAB que, segundo sua documentação, diferenças entre eles devem ser
consideradas como \ingles{bugs}.},
é totalmente compatível com o 
\href{https://www.mathworks.com/products/matlab.html}{MATLAB}~\citep{matlab2019a}\footnote{E, com a
demanda crescente para cientistas de dados hoje no país, aprender Octave ou
MATLAB pode ser um grande diferencial futuro em sua carreira\ldots}.

\item Apresentar uma aplicação prática da álgebra linear através
de um problema interessante: a previsão do tempo! Você aprenderá noções sobre o
que é um Processo de Markov e como um modelo simples de previsão do tempo
baseado em cadeias de Markov pode ser resolvido através da álgebra linear.
\end{enumerate}

Para que você aproveite ao máximo essas atividades, siga este documento exatamente
como indicado, faça as tarefas apontadas e os exercícios de fixação na ordem
em que forem apresentados.

\section{Instalação do Octave}
\label{sec:instal}

A primeira coisa que você deve fazer é instalar o GNU Octave em seu computador.
Caso você já tenha o Octave instalado, incluindo os packages extras mencionados
abaixo, vá para a próxima seção. Caso contrário, continue lendo.

Existem instaladores do Octave para Linux, Windows e Mac. Vá até o site principal
do Octave em \url{https://www.gnu.org/software/octave/} e faça o download do
instalador apropriado (sistemas Linux geralmente não precisam fazer download
do Octave, basta instalar através do gerenciador de pacotes de sua distribuição).
Em caso de dificuldade consulte a documentação
(\url{https://wiki.octave.org/Category:Installation}) ou solicite auxílio ao
monitor da disciplina.

Após a instalação terminar inicie o Octave. Você deve ver algo parecido com
a figura~\ref{fig:octave-gui}:

\begin{figure}[!h]
  \begin{center}
    \caption{Octave 5.1.0 em um computador Linux}
    \label{fig:octave-gui}
    %\fbox{
       \includegraphics[scale=0.3]{imagens/octave-gui.png}
    %}
    %\footnotesize{Fonte:~\citet[][]{citação}}
  \end{center}
\end{figure}

Uma característica interessante do Octave é que ele permite que vários outros
\ingles{packages} com funcionalidades extras possam ser adicionados à instalação
base. Visite o repositório de pacotes adicionais em
\url{https://octave.sourceforge.io/packages.php} e veja quantas funcionalidades
avançadas e específicas já estão disponíveis. Nós faremos a instalação dos
seguintes pacotes adicionais:

\begin{itemize}[noitemsep]
\item symbolic
\item general
\item optim
\item data-smoothing
\item statistics
\item image
\item io
\end{itemize}

Para instalar o pacote \ingles{symbolic}, por exemplo, devemos digitar na
janela de comando do Octave o seguinte:
\begin{tcolorbox}
\begin{lstlisting}[language=bash]
pkg install -forge symbolic
\end{lstlisting}
\end{tcolorbox}

Durante a instalação dos pacotes podem aparecer avisos diversos e \ingles{warnings}
que podem ser ignorados se, no final de tudo aparecer uma mensagem semelhante
à ``For information about changes from previous versions of the symbolic package,
run 'news symbolic''', que indica que o pacote foi instalado com sucesso.

Instale todos os pacotes da lista acima (caso um pacote seja pré-requisito para
outro, o Octave emitirá um aviso e você deverá ajustar a ordem de instalação).



%%%%%%%%%%%%%%%%%%%%%%%%%%%%%%%%%%%%%%%%%%%%%%%%%%%%%%%%%%%%%%%%%%%%%%%%%%%%%%%%
%%% Apêndices
%\appendix
%\include{apend/placeholder}


%%%%%%%%%%%%%%%%%%%%%%%%%%%%%%%%%%%%%%%%%%%%%%%%%%%%%%%%%%%%%%%%%%%%%%%%%%%%%%%%
%%% Back matter
\bibliography{utils/biblioteca}

\printindex


%%%%%%%%%%%%%%%%%%%%%%%%%%%%%%%%%%%%%%%%%%%%%%%%%%%%%%%%%%%%%%%%%%%%%%%%%%%%%%%%
%%% Termina o documento
\end{document}
