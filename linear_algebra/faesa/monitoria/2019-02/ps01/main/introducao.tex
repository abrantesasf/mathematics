\section{Introdução}
\label{sec:introd}

O \emph{Problem Set 1} (PS1) tem dois propósitos principais e bem
definidos:

\begin{enumerate}
\item Fazer com que você se
familiarize com o \href{https://www.gnu.org/software/octave/}{GNU Octave}, um
software que fornece um ambiente e uma linguagem interpretada de alto nível para
programação científica, com poderosa sintaxe orientada para matemática, cálculo
numérico, solução de problemas lineares e não lineares, extensa capacidade
gráfica e de visualização de dados~\citep{octave} e que, com raríssimas
exceções\footnote{De fato, o Octave foi criado para ser tão compatível com o
MATLAB que, segundo sua documentação, diferenças entre eles devem ser
consideradas como \ingles{bugs}.},
é totalmente compatível com o 
\href{https://www.mathworks.com/products/matlab.html}{MATLAB}~\citep{matlab2019a}\footnote{E, com a
demanda crescente para cientistas de dados hoje no país, aprender Octave ou
MATLAB pode ser um grande diferencial futuro em sua carreira\ldots}.

\item Apresentar uma aplicação prática da álgebra linear através
de um problema interessante: a previsão do tempo! Você aprenderá noções sobre o
que é um Processo de Markov e como um modelo simples de previsão do tempo
baseado em cadeias de Markov pode ser resolvido através da álgebra linear.
\end{enumerate}

Para que você aproveite ao máximo essas atividades, siga este documento exatamente
como indicado, faça as tarefas apontadas e os exercícios de fixação na ordem
em que forem apresentados.
