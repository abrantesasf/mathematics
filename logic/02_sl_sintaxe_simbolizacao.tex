%%%%%%%%%%%%%%%%%%%%%%%%%%%%%%%%%%%%%%%%%%%%%%%%%%%%%%%%%%%%%%%%%%
% Arquivo LaTeX geral para a classe Article
%
% Abrantes Araújo Silva Filho
% abrantesasf@gmail.com
% 2018-02-25


%%%%%%%%%%%%%%%%%%%%%%%%%%%%%%%%%%%%%%%%%%%%%%%%%%%%%%%%%%%%%%%%%%
%%% Configura tipo de documento e load de packages:
\RequirePackage{ifpdf}
\ifpdf
  \documentclass[pdftex,a4paper,12pt,brazil]{article} % Se tem draft é rascunho
  %\usepackage{ae}
  \usepackage[pdftex]{geometry}
  \geometry{a4paper,left=2cm,right=2cm,top=2cm,bottom=2cm}
  \usepackage[pdftex]{graphicx}
  \usepackage{setspace}
  \usepackage[T1]{fontenc}
  \usepackage[utf8]{inputenc}
  \usepackage[brazil]{babel}
  \usepackage[brazil]{varioref}
  \usepackage[pdftex,pdfpagemode=UseOutlines,bookmarks=true,%
   bookmarksopen=true,bookmarksopenlevel=5,bookmarksnumbered=true,%
   pdfstartview=FitH,hyperfootnotes=true]{hyperref}
   \hypersetup{pdfinfo={
   Author={Abrantes Ara\'{u}jo Silva Filho},
   Title={Respostas de exerc\'{i}cios selecionados de lógica matem\'{a}tica},
   Creator={pdfLaTeX},
   Producer={pdfTeX},
   CreationDate={},
   ModDate={},
   Subject={Estudo sobre l\'{o}gica matem\'{a}tica},
   Keywords={l\'{o}gica, logic, respostas, solutions},
   }}
  %\usepackage{thumbpdf}
  \hypersetup{colorlinks,%
    debug=false,%
    linkcolor=blue,%
    citecolor=blue,%
    urlcolor=blue}
  \usepackage{cleveref}
  \mathchardef\period=\mathcode`.
\else
  \documentclass[a4paper,12pt]{article}
  \usepackage[utf8]{inputenc}
  \usepackage[T1]{fontenc}
  \usepackage[brazil]{babel}
  \usepackage[dvips]{geometry}
  \usepackage[dvips]{graphicx}
  \geometry{a4paper,left=2cm,right=2cm,top=2cm,bottom=2cm}
  \usepackage{setspace}
  \usepackage{varioref}
  \usepackage{hyperref}
  \usepackage{cleveref}
\fi


%%%%%%%%%%%%%%%%%%%%%%%%%%%%%%%%%%%%%%%%%%%%%%%
%%% Configura lingua portuguesa:
%\usepackage[brazil]{babel}
%\usepackage[utf8]{inputenc}
%\usepackage[T1]{fontenc}


%%%%%%%%%%%%%%%%%%%%%%%%%%%%%%%%%%%%%%%%%%%%%%%
%%% Altera fonte padrão
% phv=Helvetica ptm=Times ppl=Palatino pbk=bookman
% pag=AdobeAvantGarde pnc=Adobe NewCenturySchoolbook
\renewcommand{\familydefault}{ppl}


%%%%%%%%%%%%%%%%%%%%%%%%%%%%%%%%%%%%%%%%%%%%%%%
%%% Configura símbolos e bibliotecas matemáticas:
\usepackage{amsmath}
\usepackage{amssymb}
\usepackage{latexsym}
\usepackage{array}
\usepackage[ruled]{algorithm}
%\usepackage{physics}
\usepackage{syllogism}
\usepackage{mathpartir}
\usepackage{siunitx}
\sisetup{group-separator = {.}}
\sisetup{group-digits = {false}}
\sisetup{output-decimal-marker = {,}}

\usepackage{listings}
\lstset{literate=
  {á}{{\'a}}1 {é}{{\'e}}1 {í}{{\'i}}1 {ó}{{\'o}}1 {ú}{{\'u}}1
  {Á}{{\'A}}1 {É}{{\'E}}1 {Í}{{\'I}}1 {Ó}{{\'O}}1 {Ú}{{\'U}}1
  {à}{{\`a}}1 {è}{{\`e}}1 {ì}{{\`i}}1 {ò}{{\`o}}1 {ù}{{\`u}}1
  {À}{{\`A}}1 {È}{{\'E}}1 {Ì}{{\`I}}1 {Ò}{{\`O}}1 {Ù}{{\`U}}1
  {ä}{{\"a}}1 {ë}{{\"e}}1 {ï}{{\"i}}1 {ö}{{\"o}}1 {ü}{{\"u}}1
  {Ä}{{\"A}}1 {Ë}{{\"E}}1 {Ï}{{\"I}}1 {Ö}{{\"O}}1 {Ü}{{\"U}}1
  {â}{{\^a}}1 {ê}{{\^e}}1 {î}{{\^i}}1 {ô}{{\^o}}1 {û}{{\^u}}1
  {Â}{{\^A}}1 {Ê}{{\^E}}1 {Î}{{\^I}}1 {Ô}{{\^O}}1 {Û}{{\^U}}1
  {œ}{{\oe}}1 {Œ}{{\OE}}1 {æ}{{\ae}}1 {Æ}{{\AE}}1 {ß}{{\ss}}1
  {ű}{{\H{u}}}1 {Ű}{{\H{U}}}1 {ő}{{\H{o}}}1 {Ő}{{\H{O}}}1
  {ç}{{\c c}}1 {Ç}{{\c C}}1 {ø}{{\o}}1 {å}{{\r a}}1 {Å}{{\r A}}1
  {€}{{\euro}}1 {£}{{\pounds}}1 {«}{{\guillemotleft}}1
  {»}{{\guillemotright}}1 {ñ}{{\~n}}1 {Ñ}{{\~N}}1 {¿}{{?`}}1
}


%%%%%%%%%%%%%%%%%%%%%%%%%%%%%%%%%%%%%%%%%%%%%%%
%%% Configura fontes e outros símbolos
\usepackage{wasysym}
\usepackage{pifont}
\usepackage{marvosym}


%%%%%%%%%%%%%%%%%%%%%%%%%%%%%%%%%%%%%%%%%%%%%%%
%%% Ativa pacote ifthen, necessário para alguns comandos
\usepackage{ifthen}


%%%%%%%%%%%%%%%%%%%%%%%%%%%%%%%%%%%%%%%%%%%%%%%
%%% Ativa suporte a cores:
\usepackage{color}
\usepackage[dvipsnames]{xcolor}
\usepackage{xparse}


%%%%%%%%%%%%%%%%%%%%%%%%%%%%%%%%%%%%%%%%%%%%%%%
%%% Ativa figuras e tabelas
\usepackage{float}
\usepackage{wrapfig}


%%%%%%%%%%%%%%%%%%%%%%%%%%%%%%%%%%%%%%%%%%%%%%%
%%% Ativa suporte ao TikZ Code
\usepackage{tikz}
\usetikzlibrary{positioning,shapes,shadows}


%%%%%%%%%%%%%%%%%%%%%%%%%%%%%%%%%%%%%%%%%%%%%%%
%%% Ativa pacote para tabelas longas e em landscape
\usepackage{array,longtable}
\usepackage{lscape}
\usepackage{array}
\usepackage{colortbl}
\newcolumntype{M}[1]{>{\centering\arraybackslash}m{#1}}
%\newcolumntype{ML}[1]{>{$}l<{$}}
%\newcolumntype{MR}[1]{>{R}r<{R}}
\newcolumntype{L}[1]{>{\arraybackslash}m{#1}}
\newcolumntype{N}{@{}m{0pt}@{}}


%%%%%%%%%%%%%%%%%%%%%%%%%%%%%%%%%%%%%%%%%%%%%%%
%%% Ativa pacote para URLs, e-mails e pathmanes:
\usepackage{url}


%%%%%%%%%%%%%%%%%%%%%%%%%%%%%%%%%%%%%%%%%%%%%%%
%%% Commando para ``italizar´´ palavras em inglês (e outras línguas!)
\newcommand{\ingles}[1]{\textit{#1}}


%%%%%%%%%%%%%%%%%%%%%%%%%%%%%%%%%%%%%%%%%%%%%%%
%%% Commando para colocar o espaço correto entre um número e sua unidade
\newcommand{\unidade}[2]{\ensuremath{#1\,\mathrm{#2}}}
\newcommand{\unidado}[2]{{#1}\,{#2}}


%%%%%%%%%%%%%%%%%%%%%%%%%%%%%%%%%%%%%%%%%%%%%%%%%%%%%%%%%%%%
%% produz ordinal masculino ou feminino dependendo do segundo
%% argumento.  Por exemplo:
%% \ordinal{1}{a} Semana
%% \ordinal{1}{o} Encontro
\newcommand{\ordinal}[2]{%
#1%
\ifthenelse{\equal{a}{#2}}%
{\textordfeminine}%
{\textordmasculine}}


%%%%%%%%%%%%%%%%%%%%%%%%%%%%%%%%%%%%%%%%%%%%%%%
%%% Ativa suporte a sublinhado:
% A opção normalem indica que ênfase será dada por itálico
% e não por sublinhado.
\usepackage[normalem]{ulem}
% O código a seguir define mais comandos para o pacote ulem, e foi retirado
% do "LaTeX demo: exemplos com LaTeXe", de Klauss Steding Jessen:
\def\dotuline{\bgroup
  \ifdim\ULdepth=\maxdimen  % Set depth based on font, if not set already
  \settodepth\ULdepth{(j}\advance\ULdepth.4pt\fi
  \markoverwith{\begingroup
  \advance\ULdepth0.08ex
  \lower\ULdepth\hbox{\kern.15em .\kern.1em}%
  \endgroup}\ULon}
\def\dashuline{\bgroup
  \ifdim\ULdepth=\maxdimen  % Set depth based on font, if not set already
  \settodepth\ULdepth{(j}\advance\ULdepth.4pt\fi
  \markoverwith{\kern.15em
  \vtop{\kern\ULdepth \hrule width .3em}%
  \kern.15em}\ULon}


%%%%%%%%%%%%%%%%%%%%%%%%%%%%%%%%%%%%%%%%%%%%%%%
%%% Ativa pacote para indentação da primeira linha de parágrafos
%\usepackage{indentfirst}


%%%%%%%%%%%%%%%%%%%%%%%%%%%%%%%%%%%%%%%%%%%%%%%
%%% Ativa pacote enumerate, extensão ao environment enumerate:
\usepackage{enumerate}


%%%%%%%%%%%%%%%%%%%%%%%%%%%%%%%%%%%%%%%%%%%%%%%
%%% environment ``Description'', similar ao environment
%%% ``description'', mas com maior controle sobre a tabulação das
%%% entradas e de suas descrições.
%%% Adaptado de um exemplo do LaTeX Companion, pg. 64.
\newlength{\myentrylen}
\newenvironment{Description}[1]%
{\list{}
  {\settowidth{\labelwidth}{\textbf{#1}}%
    \leftmargin\labelwidth\advance\leftmargin\labelsep%
    \renewcommand{\makelabel}[1]{%
      \settowidth{\myentrylen}{\textbf{##1}}%
      \ifthenelse{\lengthtest{\myentrylen > \labelwidth}}%
      {\parbox[b]{\labelwidth}%
        {\makebox[0pt][l]{\textbf{##1}}\\\mbox{}}}
      {\textbf{##1}}%
      \hfill\relax%
      }
}}
{\endlist}


%%%%%%%%%%%%%%%%%%%%%%%%%%%%%%%%%%%%%%%%%%%%%%%
%%% Comando para epígrafe em capítulos/seções (não confundir com a epígrafe geral
% da tese, definida em página isolada nos elementos pré-textuais.
\newcommand{\epigrafe}[2]{
   \vspace{-6ex}%
     {\footnotesize%
     \begin{flushright}%
     \begin{minipage}{.6\textwidth}%
     #1
     \end{minipage}\\
     \textit{#2}%
     \end{flushright}}%
   \vspace{-3ex}}


%%%%%%%%%%%%%%%%%%%%%%%%%%%%%%%%%%%%%%%%%%%%%%%
%%% Ativa pacote para controle de cabeçalhos e rodapés e configura;
% Ativa pacote:
\usepackage{fancyhdr}
% Configura estilo padrão das páginas
\pagestyle{headings}


%%%%%%%%%%%%%%%%%%%%%%%%%%%%%%%%%%%%%%%%%%%%%%%
%%% Ativa pacote para formatar os captions:
\usepackage[normal,bf]{caption}
\captionsetup[table]{font=small,skip=0pt}
\captionsetup[figure]{skip=0pt}


%%%%%%%%%%%%%%%%%%%%%%%%%%%%%%%%%%%%%%%%%%%%%%%
%%% Ativa o MakeIndex para fazer índices remissivos:
\usepackage{makeidx}
%\makeindex


%%%%%%%%%%%%%%%%%%%%%%%%%%%%%%%%%%%%%%%%%%%%%%%
%%% Ativa pacote para fazer glossário, conforme
% "LaTeX demo: exemplos com LaTeXe", de Klauss Steding Jessen.
\usepackage{makeglo}
%\makeglossary


%%%%%%%%%%%%%%%%%%%%%%%%%%%%%%%%%%%%%%%%%%%%%%%
%%% Ativa o pacote havard de referências bibliográficas e
% define um novo comando para colocar as citações em slanted:
% As opções do pacote determinam como as citações aparecem no texto.
% As seguinte opções existem:
%      default: lista a primeira completa e as subsequentes abreviadas;
%	  full: lista todas as citações completas;
%         abbr: lista todas as citações abreviadas.
% A qualquer momento o modo de citaçõa pode ser alterado com o uso do
% comando: \citationmode{}, cujo argumento é uma das opções da lista anterior.
\usepackage[default]{harvard}
% Cria comando para colocar as citações em slanted:
\newcommand{\refbib}[1]{\textsl{#1}}
% O seguinte comando configura como as referências bibliográficas serão
% formatadas. O estilo agsm é o padrão do pacote havard. Ver manual de instrução
% do pacote para maiores informações.
\bibliographystyle{agsm}
% Configura como as citações das referências apareceram no texto. O estilo agsm
% é o padrão do pacote havard. Ver manual de instrução do pacote para maiores
% informações.
\citationstyle{agsm}


%%%%%%%%%%%%%%%%%%%%%%%%%%%%%%%%%%%%%%%%%%%%%%%
%%% Comandos específicos para este documento


%%%%%%%%%%%%%%%%%%%%%%%%%%%%%%%%%%%%%%%%%%%%%%%
%%% Determina forma de hifenização de palavras quando a hifenização
%%% padrão não estiver correta
%\hyphenation{ne-nhu-ma}
\babelhyphenation[brazil]{ne-nhu-ma Git-Hub}


%%%%%%%%%%%%%%%%%%%%%%%%%%%%%%%%%%%%%%%%%%%%%%%%%%%%%%%%%%%%%%%%%%%%%%%%%%%%%%%%%%%%%%%%%%%%%%
%%%%%%%%%%%%%%%%%%%%%%%%%%%%%%%%%%%%%%%%%%%%%%%%%%%%%%%%%%%%%%%%%%%%%%%%%%%%%%%%%%%%%%%%%%%%%%
%%%%%%%%%%%%%%%%%%%%%%%%%%%%%%%%%%%%%%%%%%%%%%%%%%%%%%%%%%%%%%%%%%%%%%%%%%%%%%%%%%%%%%%%%%%%%%
%%%%%%%%%%%%%%%%%%%%%%%%%%%%%%%%%%%%%%%%%%%%%%%%%%%%%%%%%%%%%%%%%%%%%%%%%%%%%%%%%%%%%%%%%%%%%%
%%%%%%%%%%%%%%%%%%%%%%%%%%%%%%%% COMEÇA DOCUMENTO %%%%%%%%%%%%%%%%%%%%%%%%%%%%%%%%%%%%%%%%%%%%
%%%%%%%%%%%%%%%%%%%%%%%%%%%%%%%%%%%%%%%%%%%%%%%%%%%%%%%%%%%%%%%%%%%%%%%%%%%%%%%%%%%%%%%%%%%%%%
%%%%%%%%%%%%%%%%%%%%%%%%%%%%%%%%%%%%%%%%%%%%%%%%%%%%%%%%%%%%%%%%%%%%%%%%%%%%%%%%%%%%%%%%%%%%%%
%%%%%%%%%%%%%%%%%%%%%%%%%%%%%%%%%%%%%%%%%%%%%%%%%%%%%%%%%%%%%%%%%%%%%%%%%%%%%%%%%%%%%%%%%%%%%%
%%%%%%%%%%%%%%%%%%%%%%%%%%%%%%%%%%%%%%%%%%%%%%%%%%%%%%%%%%%%%%%%%%%%%%%%%%%%%%%%%%%%%%%%%%%%%%
\begin{document}
\title{Exercícios de sintaxe e simbolização\\
  (várias fontes)\\
  \ \\
--- respostas de exercícios selecionados ---}
\author{Abrantes Araújo Silva Filho}
\date{2018-03}
\maketitle
\tableofcontents
%\newpage


%%%%%%%%%%%%%%%%%%%%%%%%%%%%%%%%%%%%%%%%%%%%%%%%%%%%%%%%%%%%%%%%%%%%%%%%%%%%%%%%%%%%%%%%%%%%%%
%%%%%%%%%%%%%%%%%%%%%%%%%%%%%%%%%%%%%%%%%%%%%%%%%%%%%%%%%%%%%%%%%%%%%%%%%%%%%%%%%%%%%%%%%%%%%%
%%%%%%%%%%%%%%%%%%%%%%%%%%%%%%%%%%%%%%%%%%%%%%%%%%%%%%%%%%%%%%%%%%%%%%%%%%%%%%%%%%%%%%%%%%%%%%
%%%%%%%%%%%%%%%%%%%%%%%%%%%%%%%%%%%%%%%%%%%%%%%%%%%%%%%%%%%%%%%%%%%%%%%%%%%%%%%%%%%%%%%%%%%%%%
%%%%%%%%%%%%%%%%%%%%%%%%%%%%%%%%%%%%%%%%%%%%%%%%%%%%%%%%%%%%%%%%%%%%%%%%%%%%%%%%%%%%%%%%%%%%%%
\section{O que é este documento?} 
\label{o_que_e}
%\thispagestyle{plain}

Este documento contém as minhas respostas aos exercícios e problemas a respeito da sintaxe e
simbolização se sentenças da linguagem formal SL (Sentential Logic), presentes em três fontes:

\begin{itemize}
\item \emph{The Logic Book}, de Merrie Bergmann, James Moor e Jack Nelson (\ordinal{6}{a}
  edição);
\item \emph{Introdução à Lógica Matemática}, de Rogério Miguel Coelho (\ordinal{1}{a}
  edição);
\item \emph{Logic I (MIT 24.241)}, curso de lógica do MIT, disponível em
  \url{https://ocw.mit.edu/courses/linguistics-and-philosophy/24-241-logic-i-fall-2009/}
\end{itemize}

ATENÇÃO: não garanto que tudo aqui está correto, pelo contrário, algumas respostas expressam
minha visão particular e podem estar em desacordo com
a ``resposta padrão'' dos autores do livro ou do professor da disciplina. Também
não garanto que todos os exercícios e problemas do capítulo ou livro estarão resolvidos aqui.
De qualquer modo, caso pretenda
utilizar este documento como base para seu próprio estudo, tenha em mente o seguinte:

\begin{quote}
  \emph{Este documento é fornecido ``no estado em que se encontra'', sem garantias de qualquer
    natureza, expressas ou implícitas. Em nenhuma hipótese o autor poderá ser responsabilizado
    por qualquer problema, dano, prejuízo material ou imaterial decorrente do uso deste
    conteúdo.}
\end{quote}

Este documento (em formato PDF), o original em \LaTeX\ e outros materiais
adicionais (se necessário) estão disponíveis no seguinte
repositório GitHub: \url{https://github.com/abrantesasf/matematica} (procure pelo
diretório ``logic'').


%%%%%%%%%%%%%%%%%%%%%%%%%%%%%%%%%%%%%%%%%%%%%%%%%%%%%%%%%%%%%%%%%%%%%%%%%%%%%%%%%%%%%%%%%%%%%%
%%%%%%%%%%%%%%%%%%%%%%%%%%%%%%%%%%%%%%%%%%%%%%%%%%%%%%%%%%%%%%%%%%%%%%%%%%%%%%%%%%%%%%%%%%%%%%
%%%%%%%%%%%%%%%%%%%%%%%%%%%%%%%%%%%%%%%%%%%%%%%%%%%%%%%%%%%%%%%%%%%%%%%%%%%%%%%%%%%%%%%%%%%%%%
%%%%%%%%%%%%%%%%%%%%%%%%%%%%%%%%%%%%%%%%%%%%%%%%%%%%%%%%%%%%%%%%%%%%%%%%%%%%%%%%%%%%%%%%%%%%%%
%%%%%%%%%%%%%%%%%%%%%%%%%%%%%%%%%%%%%%%%%%%%%%%%%%%%%%%%%%%%%%%%%%%%%%%%%%%%%%%%%%%%%%%%%%%%%%
\section{Exercícios do \emph{The Logic Book}, capítulo 2}
\label{tlb-2}
%\thispagestyle{plain}


%%%%%%%%%%%%%%%%%%%%%%%%%%%%%%%%%%%%%%%%%%%%%%%%%%%%%%%%%%%%%%%%%%%%%%%%%%%%%%%%%%%%%%%%%%%%%%
%%%%%%%%%%%%%%%%%%%%%%%%%%%%%%%%%%%%%%%%%%%%%%%%%%%%%%%%%%%%%%%%%%%%%%%%%%%%%%%%%%%%%%%%%%%%%%
\subsection{Seção 2.1E}
\label{tlb-2-21e}

\paragraph{1.a)} Não é válida, pois o conectivo $\wedge$ é binário, exigindo que seja
utilizado entre dois componentes (entre 2 conjuntos).

\paragraph{1.b)} É válida pois informalmente podemos omitir os parênteses externos.

\paragraph{1.c)} É uma negação válida.

\paragraph{1.d)} Não é válida, pois a negação não é conectivo entre dois componentes.

\paragraph{1.e)} É válida.

\paragraph{1.f)} Não é válida pois \textbf{P} e \textbf{Q} são metavariáveis utilizadas
para falar sobre expressões da linguagem SL, não para uso direto como uma expressão
da linguagem SL.

\paragraph{1.g)} É válida, podemos usar colchetes no lugar dos parênteses para
facilitar o entendimento.

\paragraph{1.h)} É válida, podemos usar conjunção com negação.

\paragraph{1.i)} Não é válida, pois falta fechar um parêntese antes do colchete
de fechamento.

\paragraph{2.a)} Condição material.

\paragraph{2.b)} Disjunção.

\paragraph{2.c)} Bicondição material.

\paragraph{2.d)} Condição material.

\paragraph{2.e)} Condição material.

\paragraph{2.f)} Bicondição material.

\paragraph{2.g)} Conjunção.

\paragraph{2.h)} Negação.

\paragraph{2.i)} Condição material.

\paragraph{2.j)} Conjunção.

\paragraph{2.k)} Negação.

\paragraph{2.l)} Condição material.

\paragraph{2.m)} Disjunção

\paragraph{2.n)} Condição material.

\paragraph{3.a)} Conectivo principal: $\wedge$; componentes: $\sim A, A, H$

\paragraph{3.c)} Conectivo principal: $\vee$; componentes: $\sim (S \wedge G)$, $B$,
$(S \wedge G)$, $S$, $G$

\paragraph{3.e)} Conectivo principal: $\supset$; componentes: $(C \equiv K)$, $[\sim H \supset (M \wedge N)]$,
$C$, $K$, $\sim H$, $H$, $(M \supset N)$, $M$, $N$

\paragraph{4.a)} Não pode ocorrer imediatamente antes do conector $\sim$, e não pode aparecer
imediatamente depois de uma sentença ``A''.

\paragraph{4.c)} Pode ocorrer imediatamente antes do conector $\sim$, e não pode aparecer
imediatamente depois de uma sentença ``A''.

\paragraph{4.e)} Igual o anterior.

%%%%%%%%%%%%%%%%%%%%%%%%%%%%%%%%%%%%%%%%%%%%%%%%%%%%%%%%%%%%%%%%%%%%%%%%%%%%%%%%%%%%%%%%%%%%%%
%%%%%%%%%%%%%%%%%%%%%%%%%%%%%%%%%%%%%%%%%%%%%%%%%%%%%%%%%%%%%%%%%%%%%%%%%%%%%%%%%%%%%%%%%%%%%%
\subsection{Seção 2.2E}
\label{tlb-2-22e}

\paragraph{1.a)} Bob não é um corredor de maratona.

\begin{quote}
  \emph{Paráfrase}: ``\underline{Não é verdade que} Bob é um corredor de maratona.''

  \emph{Chave de simbolização}: ``B: Bob é um corretor de maratona.''

  \emph{Sentença simbólica}: $\sim B$
\end{quote}

\paragraph{1.b)} Alberto e Bob são jogadores.

\begin{quote}
  \emph{Paráfrase}: ``Alberto é jogador \underline{e} Bob é jogador.''

  \emph{Chave de simbolização}: ``A: Alberto é jogador; B: Bob é jogador.''

  \emph{Sentença simbólica}: $A \wedge B$
\end{quote}

\paragraph{1.c)} Se Carol é corredora, ela também é uma maratonista.

\begin{quote}
  \emph{Paráfrase}: ``\underline{Se} Carol é corredora \underline{então} Carol é maratonista.''

  \emph{Chave de simbolização}: ``C: Carol é corredora; M: Carol é maratonista.''

  \emph{Sentença simbólica}: $C \rightarrow M$
\end{quote}

\paragraph{1.d)} Alguns corredores são maratonistas.

\begin{quote}
  \emph{Paráfrase}: ``\underline{Não é verdade que} todos os corredores são maratonistas.''

  \emph{Chave de simbolização}: ``C: Todos os corredores são maratonistas.''

  \emph{Sentença simbólica}: $\sim C$
\end{quote}

\paragraph{1.e)} Carol correrá a maratona de Boston se, e apenas se, Alberto correr também.

\begin{quote}
  \emph{Paráfrase}: ``Carol correrá a maratona de Boston \underline{se, e apenas se,} Alberto também correr
  na maratona de Boston.''

  \emph{Chave de simbolização}: ``C: Carol correrá a maratona de Boston; A: Alberto correrá a maratona de Boston.''

  \emph{Sentença simbólica}: $C \Leftrightarrow A$
\end{quote}

\paragraph{1.f)} Nem todos os corredores são maratonistas.

\begin{quote}
  \emph{Paráfrase}: ``\underline{Não é verdade que} todos os corredores são maratonistas.''

  \emph{Chave de simbolização}: ``M: todos os corredores são maratonistas.''

  \emph{Sentença simbólica}: $\sim M$
\end{quote}

\paragraph{1.g)} Carol ou Alberto correrão na maratona de Boston.

\begin{quote}
  \emph{Paráfrase}: ``Carol correrá na maratona de Boston \underline{ou} Alberto correrá na maratona de Boston.''

  \emph{Chave de simbolização}: ``C: Carol correrá na maratona de Boston; A: Alberto correrá na maratona de Boston.''

  \emph{Sentença simbólica}: $C \vee A$
\end{quote}

\paragraph{1.h)} Se Carol correr na maratona de Boston, então Alberto também correrá.

\begin{quote}
  \emph{Paráfrase}: ``\underline{Se} Carol correr na maratona de Boston \underline{então} Alberto também correrá na maratona de Boston.''

  \emph{Chave de simbolização}: ``C: Carol correrá na maratona de Boston; A: Alberto correrá na maratona de Boston.''

  \emph{Sentença simbólica}: $C \rightarrow A$
\end{quote}

\paragraph{2.a)} Se Felipe tirar férias nas Bermudas, Clara tambem irá.

\begin{quote}
  \emph{Paráfrase}: ``\underline{Se} Felipe tirar férias nas Bermudas \underline{então} Clara também irá tirar férias nas Bermudas.''

  \emph{Chave de simbolização}: ``F: Felipe irá tirar férias nas Bermudas; C: Clara irá tirar férias nas Bermudas.''

  \emph{Sentença simbólica}: $F \rightarrow C$
\end{quote}

\paragraph{2.b)} Verônica irá tirar férias nas Bermudas apenas se Clara o fizer.

\begin{quote}
  \emph{Paráfrase}: ``\underline{Se} Verônica tirar férias nas Bermudas \underline{então} Clara irá tirar férias nas Bermudas.''

  \emph{Chave de simbolização}: ``V: Verônica irá tirar férias nas Bermudas; C: Clara irá tirar férias nas Bermudas.''

  \emph{Sentença simbólica}: $V \rightarrow C$
\end{quote}

\paragraph{2.c)} Verônica irá tirar férias nas Bermudas se Felipe for.

\begin{quote}
  \emph{Paráfrase}: ``\underline{Se} Felipe for tirar férias nas Bermudas \underline{então} Verônica irá tirar férias nas Bermudas.''

  \emph{Chave de simbolização}: ``F: Felipe irá tirar férias nas Bermudas; V: Verônica irá tirar férias nas Bermudas.''

  \emph{Sentença simbólica}: $F \rightarrow V$
\end{quote}

\paragraph{2.d)} Clara ou Roberto irão para as Bermudas nas férias.

\begin{quote}
  \emph{Paráfrase}: ``Clara irá para as Bermudas nas férias \underline{ou} Roberto irá para as Bermudas nas férias.''

  \emph{Chave de simbolização}: ``C: Clara irá para as Bermudas nas férias; R: Roberto irá para as Bermudas nas férias.''

  \emph{Sentença simbólica}: $C \vee R$
\end{quote}

\paragraph{2.e)} Verônica irá para as férias nas Bermudas, desde que Clara vá.

\begin{quote}
  \emph{Paráfrase}: ``\underline{Se} Clara for para as Bermundas nas férias \underline{então} Verônica irá para as Bermudas nas férias.''

  \emph{Chave de simbolização}: ``C: Clara irá para as Bermudas nas férias; V: Verônica irá para as Bermudas nas férias.''

  \emph{Sentença simbólica}: $C \rightarrow V$
\end{quote}

\paragraph{2.f)} Roberto não irá para as Bermudas nas férias.

\begin{quote}
  \emph{Paráfrase}: ``\underline{Não é verdade que} Roberto irá para as Bermudas nas férias.''

  \emph{Chave de simbolização}: ``R: Roberto irá para as Bermudas nas férias.''

  \emph{Sentença simbólica}: $\sim R$
\end{quote}




%%%%%%%%%%%%%%%%%%%%%%%%%%%%%%%%%%%%%%%%%%%%%%%%%%%%%%%%%%%%%%%%%%%%%%%%%%%%%%%%%%%%%%%%%%%%%%
%%%%%%%%%%%%%%%%%%%%%%%%%%%%%%%%%%%%%%%%%%%%%%%%%%%%%%%%%%%%%%%%%%%%%%%%%%%%%%%%%%%%%%%%%%%%%%
%%%%%%%%%%%%%%%%%%%%%%%%%%%%%%%%%%%%%%%%%%%%%%%%%%%%%%%%%%%%%%%%%%%%%%%%%%%%%%%%%%%%%%%%%%%%%%
%%%%%%%%%%%%%%%%%%%%%%%%%%%%%%%%%%%%%%%%%%%%%%%%%%%%%%%%%%%%%%%%%%%%%%%%%%%%%%%%%%%%%%%%%%%%%%
%%%%%%%%%%%%%%%%%%%%%%%%%%%%%%%%%%%%%%%%%%%%%%%%%%%%%%%%%%%%%%%%%%%%%%%%%%%%%%%%%%%%%%%%%%%%%%
\section{Exercícios do \emph{Introdução à Lógica Matemática}, capítulo 2}
\label{ilm-2}
%\thispagestyle{plain}


%%%%%%%%%%%%%%%%%%%%%%%%%%%%%%%%%%%%%%%%%%%%%%%%%%%%%%%%%%%%%%%%%%%%%%%%%%%%%%%%%%%%%%%%%%%%%%
%%%%%%%%%%%%%%%%%%%%%%%%%%%%%%%%%%%%%%%%%%%%%%%%%%%%%%%%%%%%%%%%%%%%%%%%%%%%%%%%%%%%%%%%%%%%%%
\subsection{Seção 1.6}
\label{ilm-1-16}

\paragraph{1.a)} 


%%%%%%%%%%%%%%%%%%%%%%%%%%%%%%%%%%%%%%%%%%%%%%%%%%%%%%%%%%%%%%%%%%%%%%%%%%%%%%%%%%%%%%%%%%%%%%
%%%%%%%%%%%%%%%%%%%%%%%%%%%%%%%%%%%%%%%%%%%%%%%%%%%%%%%%%%%%%%%%%%%%%%%%%%%%%%%%%%%%%%%%%%%%%%
%%%%%%%%%%%%%%%%%%%%%%%%%%%%%%%%%%%%%%%%%%%%%%%%%%%%%%%%%%%%%%%%%%%%%%%%%%%%%%%%%%%%%%%%%%%%%%
%%%%%%%%%%%%%%%%%%%%%%%%%%%%%%%%%%%%%%%%%%%%%%%%%%%%%%%%%%%%%%%%%%%%%%%%%%%%%%%%%%%%%%%%%%%%%%
%%%%%%%%%%%%%%%%%%%%%%%%%%%%%%%%%%%%%%%%%%%%%%%%%%%%%%%%%%%%%%%%%%%%%%%%%%%%%%%%%%%%%%%%%%%%%%
%\newpage
%\section{Problemas}
%\label{problemas}
%\thispagestyle{plain}


%%%%%%%%%%%%%%%%%%%%%%%%%%%%%%%%%%%%%%%%%%%%%%%%%%%%%%%%%%%%%%%%%%%%%%%%%%%%%%%%%%%%%%%%%%%%%%
%\subsection{Problema 2.1}
%\label{problema_2_1}


%%%%%%%%%%%%%%%%%%%%%%%%%%%%%%%%%%%%%%%%%%%%%%%
%%% Produz as referências bibliográficas
%% Configura título das referências bibliográficas:
%\newpage
%\renewcommand{\refname}{Referências bibliográficas}
%% Ativa arquivo com as referências bibliográficas:
%\bibliography{evandro}
%% Adiciona entrada na toc:
%\addcontentsline{toc}{section}{Referências bibliográficas}
%% Estilo da página
%\thispagestyle{headings}
%\index{referencias bibliograficas@Referências bibliográficas}


% Termina o documento
\end{document}             
