%%%%%%%%%%%%%%%%%%%%%%%%%%%%%%%%%%%%%%%%%%%%%%%%%%%%%%%%%%%%%%%%%%%%%%%%%%%%%%%%%
% Template: Article
%
% Por: Abrantes Araújo Silva Filho
%      abrantesasf@gmail.com
%
% Citação: Se você gostou deste template, por favor ajude a divulgá-lo mantendo
%          o link para meu repositório GitHub em:
%          https://github.com/abrantesasf/LaTeX
%%%%%%%%%%%%%%%%%%%%%%%%%%%%%%%%%%%%%%%%%%%%%%%%%%%%%%%%%%%%%%%%%%%%%%%%%%%%%%%%%




%%%%%%%%%%%%%%%%%%%%%%%%%%%%%%%%%%%%%%%%%%%%%%%%%%%%%%%%%%%%%%%%%%%%%%%%%%%%%%%%%
%%% Configura o tipo de documento, papel, tamanho da fonte e informações básicas
%%% para as proriedades do PDF/DVIPS e outras propriedades do documento
\RequirePackage{ifpdf}
\ifpdf
  % Classe, língua e tamanho da fonte padrão. Outras opções a considerar:
  %   draft
  %   onecolumn (padrão) ou twocolumn (OU usar o package multicol)
  %   fleqn com ou sem leqno (alinhamento à esquerda das fórmulas e dos números)
  %   oneside (padrão para article ou report) ou twoside (padrão para book)
  \documentclass[pdftex, brazil, 12pt, twoside]{article}
\else
  % Classe, língua e tamanho da fonte padrão. Outras opções a considerar:
  %   draft
  %   onecolumn (padrão) ou twocolumn (OU usar o package multicol)
  %   fleqn com ou sem leqno (alinhamento à esquerda das fórmulas e dos números)
  %   oneside (padrão para article ou report) ou twoside (padrão para book)
  \documentclass[brazil, 12pt]{article}
\fi


%%%%%%%%%%%%%%%%%%%%%%%%%%%%%%%%%%%%%%%%%%%%%%%%%%%%%%%%%%%%%%%%%%%%%%%%%%%%%%%%%
%%% Carrega pacotes iniciais necessários para estrutura de controle e para a
%%% criação e o parse de novos comandos
\usepackage{ifthen}
\usepackage{xparse}


%%%%%%%%%%%%%%%%%%%%%%%%%%%%%%%%%%%%%%%%%%%%%%%%%%%%%%%%%%%%%%%%%%%%%%%%%%%%%%%%%
%%% Configuração do tamanho da página, margens, espaçamento entrelinhas e, se
%%% necessário, ativa a indentação dos primeiros parágrafos.
\ifpdf
  \usepackage[pdftex]{geometry}
\else
  \usepackage[dvips]{geometry}
\fi
\geometry{a4paper, left=2.6cm, right=4.0cm, top=3.0cm, bottom=3.4cm}

\usepackage{setspace}
  \singlespacing
  %\onehalfspacing
  %\doublespacing


%%%%%%%%%%%%%%%%%%%%%%%%%%%%%%%%%%%%%%%%%%%%%%%%%%%%%%%%%%%%%%%%%%%%%%%%%%%%%%%%%
%%% Configurações de cabeçalho e rodapé:
\usepackage{fancyhdr}
\setlength{\headheight}{1cm}
\setlength{\footskip}{1.5cm}
\renewcommand{\headrulewidth}{0.3pt}
\renewcommand{\footrulewidth}{0.0pt}
\pagestyle{fancy}
\renewcommand{\sectionmark}[1]{%
  \markboth{\uppercase{#1}}{}}
\renewcommand{\subsectionmark}[1]{%
  \markright{\uppercase{\thesubsection \hspace{0.1cm} #1}}{}}
\fancyhead{}
\fancyfoot{}
\newcommand{\diminuifonte}{%
    \fontsize{9pt}{9}\selectfont
}
\newcommand{\aumentafonte}{%
    \fontsize{12}{12}\selectfont
}
% Configura cabeçalho e rodapé para documentos TWOSIDE
\fancyhead[EL]{\textbf{\thepage}}
\fancyhead[EC]{}
\fancyhead[ER]{\diminuifonte \textbf{\leftmark}}
\fancyhead[OR]{\textbf{\thepage}}
\fancyhead[OC]{}
\fancyhead[OL]{\diminuifonte \textbf{\rightmark}}
\fancyfoot[EL,EC,ER,OR,OC,OL]{}
% Configura cabeçalho e rodapé para documentos ONESIDE
%\lhead{ \fancyplain{}{sup esquerdo} }
%\chead{ \fancyplain{}{sup centro} }
%\rhead{ \fancyplain{}{\thesection} }
%\lfoot{ \fancyplain{}{inf esquerdo} }
%\cfoot{ \fancyplain{}{inf centro} }
%\rfoot{ \fancyplain{}{\thepage} }




%%%%%%%%%%%%%%%%%%%%%%%%%%%%%%%%%%%%%%%%%%%%%%%%%%%%%%%%%%%%%%%%%%%%%%%%%%%%%%%%%
%%% Configurações de encoding, lingua e fontes:
\usepackage[T1]{fontenc}
\usepackage[utf8]{inputenc}
\usepackage{babel}

% Altera a fonte padrão do documento (nem todas funcionam em modo math):
%   phv = Helvetica
%   ptm = Times
%   ppl = Palatino
%   pbk = bookman
%   pag = AdobeAvantGarde
%   pnc = Adobe NewCenturySchoolBook
\renewcommand{\familydefault}{ppl}


%%%%%%%%%%%%%%%%%%%%%%%%%%%%%%%%%%%%%%%%%%%%%%%%%%%%%%%%%%%%%%%%%%%%%%%%%%%%%%%%%
%%% Carrega pacotes para referências cruzadas, citações dentro do documento,
%%% links para internet e outros.Configura algumas opções.
%%% Não altere a ordem de carregamento dos packages.
\usepackage{varioref}
\ifpdf
  \usepackage[pdftex]{hyperref}
    \hypersetup{
      % Informações variáveis em cada documento (MUDE AQUI!):
      pdftitle={Calculus 1A: Differentiation},
      pdfauthor={MITx on EdX},
      pdfsubject={MITx 18.01.1x on EdX --- Calculus 1A: Differentiation},
      pdfkeywords={calculus, derivative},
      pdfinfo={
        CreationDate={}, % Ex.: D:AAAAMMDDHH24MISS
        ModDate={}       % Ex.: D:AAAAMMDDHH24MISS
      },
      % Coisas que você não deve alterar se não souber o que está fazendo:
      unicode=true,
      pdflang={pt-BR},
      bookmarksopen=true,
      bookmarksnumbered=true,
      bookmarksopenlevel=5,
      pdfdisplaydoctitle=true,
      pdfpagemode=UseOutlines,
      pdfstartview=FitH,
      pdfcreator={LaTeX with hyperref package},
      pdfproducer={pdfTeX},
      pdfnewwindow=true,
      colorlinks=true,
      citecolor=green,
      linkcolor=red,
      filecolor=cyan,
      urlcolor=blue
    }
\else
  \usepackage{hyperref}
\fi
\usepackage{cleveref}
\usepackage{url}


%%%%%%%%%%%%%%%%%%%%%%%%%%%%%%%%%%%%%%%%%%%%%%%%%%%%%%%%%%%%%%%%%%%%%%%%%%%%%%%%%
%%% Carrega bibliotecas de símbolos (matemáticos, físicos, etc.), fontes
%%% adicionais, e configura algumas opções
\usepackage{amsmath}
\usepackage{amssymb}
\usepackage{amsfonts}
\usepackage{siunitx}
  \sisetup{group-separator = {.}}
  \sisetup{group-digits = {false}}
  \sisetup{output-decimal-marker = {,}}
\usepackage{bm}
\usepackage{cancel}
% Altera separador decimal via comando, se necessário (prefira o siunitx):
%\mathchardef\period=\mathcode`.
%\DeclareMathSymbol{.}{\mathord}{letters}{"3B}
  

%%%%%%%%%%%%%%%%%%%%%%%%%%%%%%%%%%%%%%%%%%%%%%%%%%%%%%%%%%%%%%%%%%%%%%%%%%%%%%%%%
%%% Carrega packages relacionados à computação
\usepackage{algorithm2e}
\usepackage{algorithmicx}
\usepackage{algpseudocode}
\usepackage{listings}
  \lstset{literate=
    {á}{{\'a}}1 {é}{{\'e}}1 {í}{{\'i}}1 {ó}{{\'o}}1 {ú}{{\'u}}1
    {Á}{{\'A}}1 {É}{{\'E}}1 {Í}{{\'I}}1 {Ó}{{\'O}}1 {Ú}{{\'U}}1
    {à}{{\`a}}1 {è}{{\`e}}1 {ì}{{\`i}}1 {ò}{{\`o}}1 {ù}{{\`u}}1
    {À}{{\`A}}1 {È}{{\'E}}1 {Ì}{{\`I}}1 {Ò}{{\`O}}1 {Ù}{{\`U}}1
    {ä}{{\"a}}1 {ë}{{\"e}}1 {ï}{{\"i}}1 {ö}{{\"o}}1 {ü}{{\"u}}1
    {Ä}{{\"A}}1 {Ë}{{\"E}}1 {Ï}{{\"I}}1 {Ö}{{\"O}}1 {Ü}{{\"U}}1
    {â}{{\^a}}1 {ê}{{\^e}}1 {î}{{\^i}}1 {ô}{{\^o}}1 {û}{{\^u}}1
    {Â}{{\^A}}1 {Ê}{{\^E}}1 {Î}{{\^I}}1 {Ô}{{\^O}}1 {Û}{{\^U}}1
    {œ}{{\oe}}1 {Œ}{{\OE}}1 {æ}{{\ae}}1 {Æ}{{\AE}}1 {ß}{{\ss}}1
    {ű}{{\H{u}}}1 {Ű}{{\H{U}}}1 {ő}{{\H{o}}}1 {Ő}{{\H{O}}}1
    {ç}{{\c c}}1 {Ç}{{\c C}}1 {ø}{{\o}}1 {å}{{\r a}}1 {Å}{{\r A}}1
    {€}{{\euro}}1 {£}{{\pounds}}1 {«}{{\guillemotleft}}1
    {»}{{\guillemotright}}1 {ñ}{{\~n}}1 {Ñ}{{\~N}}1 {¿}{{?`}}1
  }
  

%%%%%%%%%%%%%%%%%%%%%%%%%%%%%%%%%%%%%%%%%%%%%%%%%%%%%%%%%%%%%%%%%%%%%%%%%%%%%%%%%
%%% Ativa suporte extendido a cores
\usepackage[svgnames]{xcolor} % Opções de cores: usenames (16), dvipsnames (64),
                              % svgnames (150) e x11names (300).


%%%%%%%%%%%%%%%%%%%%%%%%%%%%%%%%%%%%%%%%%%%%%%%%%%%%%%%%%%%%%%%%%%%%%%%%%%%%%%%%%
%%% Suporte à importação de gráficos externos
\ifpdf
  \usepackage[pdftex]{graphicx}
\else
  \usepackage[dvips]{graphicx}
\fi


%%%%%%%%%%%%%%%%%%%%%%%%%%%%%%%%%%%%%%%%%%%%%%%%%%%%%%%%%%%%%%%%%%%%%%%%%%%%%%%%%
%%% Suporte à criação de gráficos proceduralmente na LaTeX:
\usepackage{tikz}
  \usetikzlibrary{arrows,automata,backgrounds,matrix,patterns,positioning,shapes,shadows}


%%%%%%%%%%%%%%%%%%%%%%%%%%%%%%%%%%%%%%%%%%%%%%%%%%%%%%%%%%%%%%%%%%%%%%%%%%%%%%%%%
%%% Packages para tabelas
\usepackage{array}
\usepackage{longtable}
\usepackage{tabularx}
\usepackage{tabu}
\usepackage{lscape}
\usepackage{colortbl}  
\usepackage{booktabs}


%%%%%%%%%%%%%%%%%%%%%%%%%%%%%%%%%%%%%%%%%%%%%%%%%%%%%%%%%%%%%%%%%%%%%%%%%%%%%%%%%
%%% Packages ambientes de listas
\usepackage{enumitem}
\usepackage[ampersand]{easylist}


%%%%%%%%%%%%%%%%%%%%%%%%%%%%%%%%%%%%%%%%%%%%%%%%%%%%%%%%%%%%%%%%%%%%%%%%%%%%%%%%%
%%% Packages para suporte a ambientes floats, captions, etc.:
\usepackage{float}
\usepackage{wrapfig}
\usepackage{placeins}
\usepackage{caption}
\usepackage{sidecap}
\usepackage{subcaption}


%%%%%%%%%%%%%%%%%%%%%%%%%%%%%%%%%%%%%%%%%%%%%%%%%%%%%%%%%%%%%%%%%%%%%%%%%%%%%%%%%
%%% Meus comandos específicos:
% Commando para ``italizar´´ palavras em inglês (e outras línguas!):
\newcommand{\ingles}[1]{\textit{#1}}

% Commando para colocar o espaço correto entre um número e sua unidade:
\newcommand{\unidade}[2]{\ensuremath{#1\,\mathrm{#2}}}
\newcommand{\unidado}[2]{{#1}\,{#2}}

% Produz ordinal masculino ou feminino dependendo do segundo argumento:
\newcommand{\ordinal}[2]{%
#1%
\ifthenelse{\equal{a}{#2}}%
{\textordfeminine}%
{\textordmasculine}}


%%%%%%%%%%%%%%%%%%%%%%%%%%%%%%%%%%%%%%%%%%%%%%%%%%%%%%%%%%%%%%%%%%%%%%%%%%%%%%%%%
%%% Hifenização específica quando o LaTeX/Babel não conseguirem hifenizar:
\babelhyphenation{Git-Hub}


%%%%%%%%%%%%%%%%%%%%%%%%%%%%%%%%%%%%%%%%%%%%%%%%%%%%%%%%%%%%%%%%%%%%%%%%%%%%%%%%%
%%%%%%%%%%%%%%%%%%%%%%%%%%%%%%%%%%%%%%%%%%%%%%%%%%%%%%%%%%%%%%%%%%%%%%%%%%%%%%%%%
%%%%%%%%%%%%%%%%%%%%%%%%%%%%%%%%%%%%%%%%%%%%%%%%%%%%%%%%%%%%%%%%%%%%%%%%%%%%%%%%%
%%%%%%%%%%%%%%%%%%%%%%%%%%%%%%%%%%%%%%%%%%%%%%%%%%%%%%%%%%%%%%%%%%%%%%%%%%%%%%%%%
%%%%%%%%%%%%%%%%%%%%%%%%%%%%%% COMEÇA O DOCUMENTO %%%%%%%%%%%%%%%%%%%%%%%%%%%%%%%
%%%%%%%%%%%%%%%%%%%%%%%%%%%%%%%%%%%%%%%%%%%%%%%%%%%%%%%%%%%%%%%%%%%%%%%%%%%%%%%%%
%%%%%%%%%%%%%%%%%%%%%%%%%%%%%%%%%%%%%%%%%%%%%%%%%%%%%%%%%%%%%%%%%%%%%%%%%%%%%%%%%
%%%%%%%%%%%%%%%%%%%%%%%%%%%%%%%%%%%%%%%%%%%%%%%%%%%%%%%%%%%%%%%%%%%%%%%%%%%%%%%%%
%%%%%%%%%%%%%%%%%%%%%%%%%%%%%%%%%%%%%%%%%%%%%%%%%%%%%%%%%%%%%%%%%%%%%%%%%%%%%%%%%
\begin{document}
\title{Calculus 1A: Differentiation}
\author{MITx 18.01.1x}
\date{17/08/2018 -- 20/11/2018}
\maketitle
\tableofcontents
\newpage


%%%%%%%%%%%%%%%%%%%%%%%%%%%%%%%%%%%%%%%%%%%%%%%%%%%%%%%%%%%%%%%%%%%%%%%%%%%%%%%%%
%%%%%%%%%%%%%%%%%%%%%%%%%%%%%%%%%%%%%%%%%%%%%%%%%%%%%%%%%%%%%%%%%%%%%%%%%%%%%%%%%
%%%%%%%%%%%%%%%%%%%%%%%%%%%%%%%%%%%%%%%%%%%%%%%%%%%%%%%%%%%%%%%%%%%%%%%%%%%%%%%%%
\section{Getting started (2018--08--17)}
\label{gs}


%%%%%%%%%%%%%%%%%%%%%%%%%%%%%%%%%%%%%%%%%%%%%%%%%%%%%%%%%%%%%%%%%%%%%%%%%%%%%%%%%
%%%%%%%%%%%%%%%%%%%%%%%%%%%%%%%%%%%%%%%%%%%%%%%%%%%%%%%%%%%%%%%%%%%%%%%%%%%%%%%%%
\subsection{Overview and logistics}
\label{gs-ol}

%%%%%%%%%%%%%%%%%%%%%%%%%%%%%%%%%%%%%%%%%%%%%%%%%%%%%%%%%%%%%%%%%%%%%%%%%%%%%%%%%
\subsubsection{Meet the course team}
\label{gs-ol-team}

\paragraph{Professor David Jerison}
David Jerison received his Ph.D.\ from Princeton University in 1980, and joined the mathematics faculty at MIT in 1981. In 1985, he received an A.P.\ Sloan Foundation Fellowship and a Presidential Young Investigator Award. In 1999 he was elected to the American Academy of Arts and Sciences. In 2004, he was selected for a Margaret MacVicar Faculty Fellowship in recognition of his teaching. In 2012, the American Mathematical Society awarded him and his collaborator Jack Lee the Bergman Prize in Complex Analysis.

\begin{figure}[H]
  \begin{center}
    \caption{Professor David Jerison}
    \label{fig:david-jerison}
    \fbox{\includegraphics[scale=0.7]{imagens/jerison.jpg}}
    %\includegraphics[scale=0.4]{imagens/palavras-imagens-simbolos.png}
    %
    %\footnotesize{Fonte:}
  \end{center}
\end{figure}

Professor Jerison's research focuses on PDEs and Fourier Analysis. He has taught single variable calculus, multivariable calculus, and differential equations at MIT several times each.

\paragraph{Professor Gigliola Staffilan}
Gigliola Staffilani is the Abby Rockefeller Mauzé Professor of Mathematics since 2007. She received her Ph.D.\ from the University of Chicago in 1995. Following faculty appointments at Stanford, Princeton, and Brown, she joined the MIT mathematics faculty in 2002. She received both a teaching award and a research fellowship while at Stanford. She received a Sloan Foundation Fellowship in 2000. In 2014 she was elected to the American Academy of Arts and Sciences.

\begin{figure}[H]
  \begin{center}
    \caption{Professor Gigliola Staffilani}
    \label{fig:gigliola}
    \fbox{\includegraphics[scale=0.7]{imagens/staffilani.jpg}}
    %\includegraphics[scale=0.4]{imagens/palavras-imagens-simbolos.png}
    %
    %\footnotesize{Fonte:}
  \end{center}
\end{figure}

Professor Staffilani is an analyst, with a concentration on dispersive nonlinear PDEs. She has taught multivariable calculus several times at MIT, as well as differential equations.

\paragraph{Instructor Jen French}
Jen French is an MITx Digital Learning Scientist in the MIT math department. She earned her Ph.D.\ in mathematics from MIT in 2010, with specialization in Algebraic Topology. After teaching after school math for elementary aged students and working with the Teaching and Learning Lab at MIT developing interdisciplinary curricular videos tying foundational concepts in math and science to engineering design themes, she joined MITx in 2013. She has developed videos, visual interactives, and problems providing immediate feedback using the edX platform residentially in the MIT math department to aid student learning. She has developed the calculus series (3 courses) and differential equations series (5 courses) available here on edX.

\begin{figure}[H]
  \begin{center}
    \caption{Instructor Jen French }
    \label{fig:jen}
    \fbox{\includegraphics[scale=0.7]{imagens/french.jpg}}
    %\includegraphics[scale=0.4]{imagens/palavras-imagens-simbolos.png}
    %
    %\footnotesize{Fonte:}
  \end{center}
\end{figure}

\paragraph{Instructor Stephen Wang}
Stephen Wang earned a Ph.D.\ in mathematics from the University of Chicago in 2006, where he specialized in geometry. He has earned teaching awards from both Chicago and Harvard University, and has also been a faculty member at Haverford College and Bucknell University before jumping on board the calculus team at MIT. In fall 2015 he joined the Rice University mathematics faculty.

\begin{figure}[H]
  \begin{center}
    \caption{Instructor Stephen Wang}
    \label{fig:wang}
    \fbox{\includegraphics[scale=0.7]{imagens/wang.jpg}}
  \end{center}
\end{figure}

\paragraph{Special thanks to \ldots}
Huge thanks to Prof. Arthur Mattuck for starting it all. Big thanks to Timothy Hall for asking David Jerison the question, how do ziplines behave mathematically. We also thank David Custer and Susan Ruff who helped with real life ziplines and shared MIT student experiments on ziplines.

Ed Tech Developers:
\begin{itemize}[noitemsep]
\item J.\ M.\ Claus
\item Brian French
\item Eric Heubel
\item Haynes Miller
\item Martin Segado
\end{itemize}

MIT Undergrads:
\begin{itemize}[noitemsep]
\item Phillip Ai
\item Emanuele Ceccarelli
\item Peter Haine
\item Peter Kleinhenz
\item Mohammed Kane
\end{itemize}

MIT PhD Students:
\begin{itemize}[noitemsep]
\item Tudor Cristea-Platon
\item Kristin Kurianski
\item Lucas Tambasco
\end{itemize}

MITx Video Team:
\begin{itemize}[noitemsep]
\item Brittany Bellamy
\item Chris Boebel
\item Kenny Caudill
\item Tsinu Heramo
\item Jess Kloss
\item Douglass McLean
\item Lana Scott
\item Catilin Stier
\end{itemize}

MITx Support Staff:
\begin{itemize}[noitemsep]
\item Kyle Boots
\item Brad K.\ Goodman
\end{itemize}

\fbox{\begin{minipage}{10cm}This course was funded in part by:\ \\
Class of 1960 Alumni Funds\ \\
2014--2015 Alumni Class Funds Grant\ \\
Wertheimer Fund\end{minipage}}

%%%%%%%%%%%%%%%%%%%%%%%%%%%%%%%%%%%%%%%%%%%%%%%%%%%%%%%%%%%%%%%%%%%%%%%%%%%%%%%%%
\subsubsection{Course description}
\label{gs-ol-description}

Discover the derivative --- what it is, how to compute it, and when to apply it in solving real world problems. Part 1 of 3.

How does the final velocity on a zip line change when the starting point is raised or lowered by a matter of centimeters? What is the accuracy of a GPS position measurement? How fast should an airplane travel to minimize fuel consumption? The answers to all of these questions involve the derivative.

But what is the derivative? You will learn its mathematical notation, physical meaning, geometric interpretation, and be able to move fluently between these representations of the derivative. You will discover how to differentiate any function you can think up, and develop a powerful intuition to be able to sketch the graph of many functions. You will make linear and quadratic approximations of functions to simplify computations and gain intuition for system behavior. You will learn to maximize and minimize functions to optimize properties like cost, efficiency, energy, and power.

This course, in combination with \emph{18.01.2x Calculus 1B: Integration}, covers the AP Calculus AB curriculum.

This course, in combination with \emph{18.01.2x Calculus 1B: Integration} and \emph{18.01.3x Calculus 1C: Coordinate Systems and Infinite Series}, covers the AP Calculus BC curriculum.

If you intend to take an AP exam, we strongly suggest that you familiarize yourself with the AP exam to prepare for it.

%%%%%%%%%%%%%%%%%%%%%%%%%%%%%%%%%%%%%%%%%%%%%%%%%%%%%%%%%%%%%%%%%%%%%%%%%%%%%%%%%
\subsubsection{The making of this course}
\label{gs-ol-making}

This course was created using latex2edX, a free tool developed at MIT for creating content for edX written in \LaTeX. \LaTeX\ is a typesetting language that is fantastic for writing math! Occasionally, the equations you see in the webpage (which are rendered in mathjax) do not load appropriately. Our apologies. The easiest fix is to simply reload the page. Another solution is to change browsers. (Firefox seems to render mathjax less reliably than Chrome or Safari. However, frequent changes to edX will cause disruptions in our content.)

Note edX is not supported on tablet devices. That said, users report that 95\% of the problems can be done on a tablet device, but if weird errors are creeping in (especially with formula input type problems) you may try switching to a laptop or desktop computer.

\begin{figure}[H]
  \begin{center}
    %\caption{}
    \label{fig:latex2edx}
    \fbox{\includegraphics[scale=0.7]{imagens/latex2edx.png}}
  \end{center}
\end{figure}
 

%%%%%%%%%%%%%%%%%%%%%%%%%%%%%%%%%%%%%%%%%%%%%%%%%%%%%%%%%%%%%%%%%%%%%%%%%%%%%%%%%
\subsubsection{How to succeed}
\label{gs-ol-succeed}

\paragraph{Prerequisites}
This course has a global audience with students from a wide variety of backgrounds. To succeed in this course, you must have a solid foundation in

\begin{enumerate}[noitemsep]
\item Algebra
\item Geometry
\item Trigonometry
\item Exponents
\item Logarithms
\item Limits
\end{enumerate}

We know that many students may not have solid foundation in limits, so we have included an optional Unit 0 that introduces Limits. Understanding limits is essential to understand the first lecture on the definition of the derivative in Unit 1, some Homework problems on Continuity and Differentiability in Unit 1, and the first lecture on Limiting behavior and sketching functions in Unit 4.

Because we know you come from different backgrounds, we want to help you to choose the best path through this content. To aid us in this, please take the ``Choose your calculus adventure'' diagnostics. This will help you to determine if you have the skills to succeed, what skills you may need to review, and which units you may be able to skip!

\paragraph{Reference materials}
The material we provide in the Courseware contains all of the content you need for this course. However, there are many good calculus texts that have a great deal of problems and alternate explanations that may help you. Most widely used calculus texts are adequate.

There is also the free web resource \href{https://www.khanacademy.org/}{Khan Academy}.
Links to other web resources can be found on the Course Info page under the header ``Related Links''. Feel free to share other resources on the course wiki or through the discussion forum.


%%%%%%%%%%%%%%%%%%%%%%%%%%%%%%%%%%%%%%%%%%%%%%%%%%%%%%%%%%%%%%%%%%%%%%%%%%%%%%%%%
\subsubsection{Grading}
\label{gs-ol-grading}

There are 4 categories of graded problems in 18.01.1x: in-lecture Exercises, Part A Homework, Part B Homework, and the Final Exam.

\begin{itemize}
\item \textbf{Exercises:} These are the problems that are interspersed between videos in each lecture. These problems count for 20\% of your grade. These problems will be used to motivate theory, practice a concept you just learned, and review material from previous sequences that we are using. While you are graded on these problems, they are low-stakes: you have multiple attempts, and have the opportunity to look at the answer after you have submitted a correct answer or run out of attempts. This is where you will do the majority of your learning. We encourage you to make mistakes and learn from them!
\item \textbf{Part A Homework:} Each unit has 1 Part A Homework assignment, which gives you an opportunity to practice what you learned. These problems count for 10\% of your total grade. Wait until the end of the unit to attempt these problems. These problems help you identify the concepts that you have forgotten, and aid in long-term retention. These problems are mostly mechanical–asking you to practice methods, and techniques learned in each unit. Each problem typically tests knowledge from only one section in a unit. (We won't necessarily tell you which one though!)
\item \textbf{Part B Homework:} Each unit has 1 Part B Homework assignment. The part B homework counts for 25\% of your total grade. The problems on this homework combine ideas from all of the sequences in the unit. These problems are mostly in the form of word problems which ask you to apply the methods learned to new scenarios.
\item \textbf{Final:} The final exam is the culmination of your learning, and will account for 45\% of your grade. These problems cover all of the material in this course. Several of the problems follow the AP short-answer format. However, we cannot grade the justifications to your reasoning here. To prepare for the AP exam, you should take and review the solutions to sample AP exams from the AP website. 
\end{itemize}

Note: Please notice that Unit 0 is optional and the exercises and homework are intended for self study only and do not count towards your grade.

\paragraph{Certification} To earn an ID verified certificate, you must earn 60\% of the points in this course. You can see your progress towards certification by clicking on the Progress link above.


%%%%%%%%%%%%%%%%%%%%%%%%%%%%%%%%%%%%%%%%%%%%%%%%%%%%%%%%%%%%%%%%%%%%%%%%%%%%%%%%%
%%%%%%%%%%%%%%%%%%%%%%%%%%%%%%%%%%%%%%%%%%%%%%%%%%%%%%%%%%%%%%%%%%%%%%%%%%%%%%%%%
\subsection{Using the EdX platform}
\label{gs-edx}

%%%%%%%%%%%%%%%%%%%%%%%%%%%%%%%%%%%%%%%%%%%%%%%%%%%%%%%%%%%%%%%%%%%%%%%%%%%%%%%%%
\subsubsection{Navigating EdX}
\label{gs-edx-nav}

This course was developed at MIT and is made available to you by the edX platform.The edX platform is a platform for learning! It allows people from around the world to access content for free, based on their own interests and background.

If you have never taken a course on edX, please take the short 1 hour course
\href{https://www.edx.org/course/demox-edx-demox-1-0}{DemoX} to familiarize yourself with the platform and its capabilities.

In this course, we have the following top-level resources:

\begin{itemize}[noitemsep]
\item \textbf{Course:} This is the graded content of this course, as well as all learning materials.
\item \textbf{Calendar:} All of the due dates are in UTC, and are available in the google calendar,
  which you can download into your own calendar so that you can have these due dates available in your own time zone.
\item \textbf{Discussion:} This is a link to the full discussion forum. For specific discussions
  related to a problem or video, link through the discussion forum link at the bottom of each page.
  (See the discussion at the bottom of this page for help with these problems.)
\item \textbf{Progress:} Use this tab to see how your are progressing through the content!
\end{itemize}

\textbf{Course} is where you will spend most of your time. This is where we put the content and assessments for your learning. Everything else is a resource to support your learning.

%%%%%%%%%%%%%%%%%%%%%%%%%%%%%%%%%%%%%%%%%%%%%%%%%%%%%%%%%%%%%%%%%%%%%%%%%%%%%%%%%
\subsubsection{Example problem types}
\label{gs-edx-example}

Take a moment to familiarize yourself with the main problem types we use in this course.

\paragraph{Checking and submitting an answer:}
The edX platform is able to check your answers and give you immediate feedback. When you ``check'' a problem, it is automatically submitted for grading purposes. Depending on the type of the problem you may have access to the ``show answer'' button. In the lecture exercises as well as the part A and part B Homework assignments, this option to show the answer will appear only after the due date has passed, you have run out of problem attempts, or you have already submitted the correct answer. You will never get detailed solutions to the final exam.
Example: \emph{This problem has unlimited attempts. If you get an answer wrong, you can simply try again until you get it right. How many weeks will this course be?}

\paragraph{Resetting a Problem:}
Some problems involve randomized parameters, or other elements that you may wish to reset to the original configuration. Here is an example where the variables and are randomized. After one attempt, you can click reset to see the values change!
Example: \emph{Let $x_1 = 5$ and $x_2 = 65$. Enter the numeric value of in the answer box below.}

\paragraph{Limited Number of Attempts 1:}
Most of the time, you will have a limited number of times that you can attempt a problem. To save an answer and keep it there until you come back, use the save button.
Example: \emph{How much does it cost to take an edX course?}

\paragraph{Limited Number of Attempts 2:}
Multiple choice problems will almost always have between 1 and 3 attempts.
Example: \emph{Which choice is correct?}

\paragraph{Formula Entry Problems:}
This is a math class, which means we are going to be using formulas. And sometimes, we want you to find these formulas. There are some rules for entering formulas into the text entry box (which follows rules for ASCII math). Use:

\begin{itemize}[noitemsep]
\item $+$ to denote addition; e.g. $2+3$
\item $-$ to denote subtraction; e.g. $x-1$
\item $*$ to denote multiplication; e.g. $2*x$
\item $\wedge$ to denote exponentiation; e.g. $x \wedge 2$ for $x^2$
\item / to denote division; e.g. $7/x$ for $\frac{7}{x}$
\item ``pi'' for the mathematical constant $\pi$
\item ``e'' for the mathematical constant $e$
\item sqrt(x), sin(x), cos(x), ln(x), arccos(x), etc. for the known functions $\sqrt{x}$, $\sin{x}$, $\cos{x}$, $\ln{x}$, etc. Note that parentheses are required.
\item Use parentheses ( ) to specify order of operations.
\end{itemize}

Each formula entry box will have a Formula Input Help button below the answer button, where you can find these facts about how to enter formulas. (See the button below.) Example: \emph{enter the function $2e^{x-1} + \sqrt{y}$ using the rules above. (Type 2 * e \textasciicircum (x-1) + sqrt(y) into the answer box.)}

\paragraph{Drag and Drop Problems:}
Example: \emph{Drag and drop the elements to create the quadratic formula}. 
Use the arrows on the horizontal bar to see more options to drag into the formula.

\paragraph{Sketch Input Problems:}
We created this sketch input problem type because being able to sketch functions to reason through problems is a big part of applying calculus as a problem-solving tool. Example: \emph{Try drawing a smiley face. The mouth should lie below the x-axis, and the place an eye at the points and $(-1, 2)$ and $(1, 2)$}


%%%%%%%%%%%%%%%%%%%%%%%%%%%%%%%%%%%%%%%%%%%%%%%%%%%%%%%%%%%%%%%%%%%%%%%%%%%%%%%%%
%%%%%%%%%%%%%%%%%%%%%%%%%%%%%%%%%%%%%%%%%%%%%%%%%%%%%%%%%%%%%%%%%%%%%%%%%%%%%%%%%
\subsection{Using the forum}
\label{gs-forum}

%%%%%%%%%%%%%%%%%%%%%%%%%%%%%%%%%%%%%%%%%%%%%%%%%%%%%%%%%%%%%%%%%%%%%%%%%%%%%%%%%
\subsubsection{Discussion forum}
\label{gs-forum-forum}

The discussion forum is the tool for connecting with the community of online learners in this course. Use the forum to ask questions, seek clarifications, report bugs, start or respond to topical discussions.

On most pages, there is a link at the bottom, which says ``show discussion''. Clicking this link will show the discussion forum associated with the videos and problems on that page.

\paragraph{``Netiquette'': What to do}

\begin{itemize}[noitemsep]
\item \textbf{Be polite.} Make sure that your posts are respectful of the other students and staff in the course.
\item Use the search button. Search for similar forum posts \textbf{before you post} using the magnifying glass icon. Many of your classmates will have the same question that you do! If you perform a search first, you may find the question and answer without needing to post yourself. This helps us keep the forum organized and useful!
\item Reply to existing discussions when you see someone with the same question. This helps to organize responses.
\item Use a descriptive and specific title to your post. This will attract the attention of TAs and classmates who can answer your question.
\item Be very specific about where you need help. Are you stuck on a particular part of a problem? Are you confused by a particular concept? What have you done so far?
\item Actively up-vote other posts, and other students will up-vote yours! The more up-votes your post has, the more likely they are to be seen.
\end{itemize}

\paragraph{``Netiquette'': What not to do}
Follow common writing practices for online communication:

\begin{itemize}[noitemsep]
\item Avoid TYPING IN ALL CAPS. Some people read this as shouting, even if that is not your intention.
\item Avoid \textbf{typing in bold}. Some people read this as shouting, even if that is not your intention.
\item Avoid unnecessary symbols, abbreviated words, texting shorthand, and replacing words with numbers (e.g. Pls don't rplce wrds w/\#s).
\item Avoid repeating letters or reeeeepeeaattinggggg chaaracterrrss.
\item Avoid excessive punctuation!!!!!!!!
\end{itemize}

\paragraph{Cheating!}
We encourage you to communicate in the forum about problems, and get hints and help understanding the material from your fellow classmates and the course TAs. However:

\begin{itemize}[noitemsep]
\item Please do not post solutions to lecture problems, homework problems (part A or part B), or final exam problems. These will be removed, and the student who posted will be contacted and dealt with individually.
\item Do not post or copy solutions posted to the forum for any exercises. This is cheating.
\item Do not copy solutions from yourself. This is cheating.
\end{itemize}


%%%%%%%%%%%%%%%%%%%%%%%%%%%%%%%%%%%%%%%%%%%%%%%%%%%%%%%%%%%%%%%%%%%%%%%%%%%%%%%%%
%%%%%%%%%%%%%%%%%%%%%%%%%%%%%%%%%%%%%%%%%%%%%%%%%%%%%%%%%%%%%%%%%%%%%%%%%%%%%%%%%
\subsection{Choose your calculus adventure}
\label{gs-adventure}

%%%%%%%%%%%%%%%%%%%%%%%%%%%%%%%%%%%%%%%%%%%%%%%%%%%%%%%%%%%%%%%%%%%%%%%%%%%%%%%%%
\subsubsection{Choose your own calculus adventure}
\label{gs-adventure-choose}

You are interested in learning calculus, but we don't know very much about you or what you already know. So to help you learn best, please take the following diagnostics. These diagnostics will help you choose a path through the content that makes the most sense for you.

We want you to succeed, so make sure that you have the basic precalculus skills so you won't be frustrated! The first 4 pages test your readiness for this calculus class. If you aren't ready yet, don't worry, you can take this course later.

Some of you may already know a lot of calculus. To help you get started in the right place, we have further diagnostics. On pages 5 and 6, you can take the limits and derivatives diagnostics. We encourage you to take a look at these even if you don't know any calculus yet. But don't worry; we designed this course for people with varying backgrounds, including those with no calculus experience.


%%%%%%%%%%%%%%%%%%%%%%%%%%%%%%%%%%%%%%%%%%%%%%%%%%%%%%%%%%%%%%%%%%%%%%%%%%%%%%%%%
\subsubsection{Algebra Problems}
\label{gs-adventure-algebra}

\paragraph{A1:} What is the slope of the line through the points $(3, -5)$ and $(1, -1)$?

\paragraph{A2:} The lines $3x + 27 = 7$ and $x - 3y = 6$ intersect in a point with what coordinates?

\paragraph{A3:} Which expression is equivalent to $\displaystyle \left(\frac{1}{x}+\frac{1}{y}\right)^{-1}$?

\paragraph{A4:} List all possible solutions to the equation $x^3 - x^2 - 2x = 0$ (Use decimals only, not fractions,
and separate answers with commas.)

\paragraph{A5:} A $0.25$ mL sample of water drawn from a $5$ liter flask contains $1.25 \times 10^8$ bacteria. Give the
approximate number of bacteria in the flask, expressing your answer in scientific notation.
(Scientific Notation: find a real number $a$
between $1$ and $10$, and an integer $n$, such that $x = a \times 10^n$.)

\paragraph{A6:} For what value of the constant $a$ will the system of linear equations have no solution?

\begin{equation}
  \begin{split}
    6x - 5y &= 3\\
    3x + ay &= 1
  \end{split}
\end{equation}

\paragraph{A7:} Find the value of the constant $a$ for which the polynomial $x^3 + ax^2 -1$ will have $-1$ as a zero.

\paragraph{A8:} If $\displaystyle a_n = \frac{x^n}{2^nn!}$, find $\displaystyle \frac{a_{n+1}}{a_n}$.
\ \\

If you score $0.6$ or above, you have a good grasp of algebraic manipulations, and can do them accurately enough to succeed in this class!
Otherwise, this course will be very difficult for you. We recommend taking an algebra and/or trigonometry class to solidify your familiarity and accuracy before attempting this course.

%%%%%%%%%%%%%%%%%%%%%%%%%%%%%%%%%%%%%%%%%%%%%%%%%%%%%%%%%%%%%%%%%%%%%%%%%%%%%%%%%
\subsubsection{Geometry}
\label{gs-adventure-geometry}

\paragraph{G1:} A bed that is $4$ feet wide must enter through a door along the $8$ foot wall of a $8$ by $20$ foot room.
What is the largest length of a bed that can be rotated to fit into the position shown by the dotted lines against the back wall?

\begin{figure}[H]
  \begin{center}
    %\caption{}
    \label{fig:adv-g1}
    \fbox{\includegraphics[scale=0.5]{imagens/adventure_g1.png}}
  \end{center}
\end{figure}

\paragraph{G2:} The four-sided solid shown is the part of the solid sphere (of radius 2, centered at the origin) in the first octant. Find its total surface area.

\begin{figure}[H]
  \begin{center}
    %\caption{}
    \label{fig:adv-g2}
    \fbox{\includegraphics[scale=0.5]{imagens/adventure_g2.png}}
  \end{center}
\end{figure}

\paragraph{G3:} To estimate the height of a skyscraper 1km in the distance, Jenny finds that if her friend Steve stands 2.5 meters away, the top of his head just lines up with the top of the building. Steve is 2 meters tall, and Jenny's eye is 1.5 meters from the ground. How high is the building? (The dotted lines may help you.)

\begin{figure}[H]
  \begin{center}
    %\caption{}
    \label{fig:adv-g3}
    \fbox{\includegraphics[scale=0.5]{imagens/adventure_g3.png}}
  \end{center}
\end{figure}

%%%%%%%%%%%%%%%%%%%%%%%%%%%%%%%%%%%%%%%%%%%%%%%%%%%%%%%%%%%%%%%%%%%%%%%%%%%%%%%%%
\subsubsection{Trigonometry}
\label{gs-adventure-trigonometry}

\paragraph{T1:} In the given right triangle, what is $\tan{y}$?

\begin{figure}[H]
  \begin{center}
    %\caption{}
    \label{fig:adv-t1}
    \fbox{\includegraphics[scale=0.5]{imagens/adventure_t1.png}}
  \end{center}
\end{figure}

\paragraph{T2:} A horse runs counterclockwise (anticlockwise) around the circular track of radius 400m at a constant speed, starting at the marked point. It completes one lap in three minutes. What is its coordinate after one minute? (If needed, you can use ``pi'' for $\pi$, and sqrt(5) for $\sqrt{5}$.)

\begin{figure}[H]
  \begin{center}
    %\caption{}
    \label{fig:adv-t2}
    \fbox{\includegraphics[scale=0.5]{imagens/adventure_t2.png}}
  \end{center}
\end{figure}

\paragraph{T3:} Find the smallest positive solution to the equation $\sin{2x} = \frac{1}{2}$; here $x$ is in radians.
(If needed, you can use ``pi'' for $\pi$, and sqrt(5) for $\sqrt{5}$. You can enter fractions using the forward slash
/ ; e.g. pi/2 for $\frac{\pi}{2}$.)

\paragraph{T4:} A line with slope 1/2 makes an acute angle $\theta$ with the axis. What is $\sin{\theta}$?
(If needed, you can use pi for $\pi$, and sqrt(5) for $\sqrt{5}$. You may enter your answer as a decimal number.)

\paragraph{T5:} By using the trigonometric identity $\cos{2x} = \cos^2{x} - \sin^2{x}$, and other identities,
find the \textbf{positive} expression for $\sin{\left(\frac{A}{2}\right)}$ in terms of $\cos{A}$.

\paragraph{T6:} The graph below represents which of these functions?

\begin{figure}[H]
  \begin{center}
    %\caption{}
    \label{fig:adv-t3}
    \fbox{\includegraphics[scale=0.5]{imagens/adventure_t3.png}}
  \end{center}
\end{figure}

\begin{itemize}[noitemsep]
\item[$\square$] $\sin{x}$
\item[$\square$] $\cos{x}$
\item[$\square$] $\sin{(x/2)}$
\item[$\square$] $\cos{(x/2)}$
\item[$\square$] $\sin{2x}$
\item[$\square$] $\cos{2x}$
\end{itemize}

If you got a 0.6 or above, you have the foundational trigonometry understanding to succeed in this course.
Otherwise, you will need to study trigonometry concurrent with this course in order to succeed!

%%%%%%%%%%%%%%%%%%%%%%%%%%%%%%%%%%%%%%%%%%%%%%%%%%%%%%%%%%%%%%%%%%%%%%%%%%%%%%%%%
\subsubsection{Logarithms and exponentials}
\label{gs-adventure-log}

\paragraph{E1:} If $\log_{10}{a}=4.2$ and $\log_{10}{b} = 0.5$, what is $\log_{10}{ab}$?

\paragraph{E2:} If $2^a = \frac{\sqrt{8}}{4^3}$, what is $a$?

\paragraph{E3:} Which of the following is equal to $\sqrt{\frac{x^{16}(1 + x^2)}{9}}$?

\begin{itemize}
\item[$\square$] $\frac{x^4(1+x)}{3}$
\item[$\square$] $\frac{x^8(1+3)}{3}$
\item[$\square$] $\frac{x^4(1+x^2)^{0.5}}{3}$
\item[$\square$] $\frac{x^8(1+x^2)^{0.5}}{3}$
\item[$\square$] $\frac{x^4(1+x^2)}{3}$
\item[$\square$] $\frac{x^8(1+x^2)}{3}$
\item[$\square$] None of the above
\end{itemize}

\paragraph{E4:} Solve for $x$: $\log_{10}\left[(x+1)^2\right] = 2$. (Enter your answer as a list of $x$-values, separated by commas.)

\paragraph{E5:} A pot of water (at sea level) is boiling; the heat is turned off at time $t=0$, and $2$ minutes later the water temperature has fallen to $80$ºC. If the temperature $T$ (in ºC) is expressed in terms of time $t$ (in minutes) by the law

\begin{equation}
  T = Ae^{-kt}
\end{equation}
\noindent
find the values of the constants $A$ and $k$.

If you got a 0.7 or higher, congratulations! You have an excellent understanding of logarithms and exponents! Good work.
You can still succeed in this course if you got a 0.7 or lower, but we strongly recommend that you review logarithms and exponents before you get to the end of Unit 2: Differentiation where logarithms and exponents begin to take on a prominent role in the course.

%%%%%%%%%%%%%%%%%%%%%%%%%%%%%%%%%%%%%%%%%%%%%%%%%%%%%%%%%%%%%%%%%%%%%%%%%%%%%%%%%
\subsubsection{Limits diagnostic}
\label{gs-adventure-limits}

You will NOT see which problems you get correct and incorrect. Sorry if this is frustrating. We will be using these problems to assess whether or not our content teaches you about limits.

\paragraph{L1:} What is $\displaystyle \lim_{x\to\infty} \frac{x^2-4}{2 + x - 4x^2}$?

\begin{itemize}
\item[$\square$] $-2$
\item[$\square$] $-\frac{1}{4}$
\item[$\square$] $\frac{1}{2}$
\item[$\square$] $1$
\item[$\square$] The limit does not exist.
\end{itemize}

\paragraph{L2:} The graph of a function $f$ is shown below. If the limit as $x\to\infty$ exists and $f$ is not continuous at $b$, then $b=$?

\begin{figure}[H]
  \begin{center}
    %\caption{}
    \label{fig:adv-l2}
    \fbox{\includegraphics[scale=0.5]{imagens/adventure_l2.png}}
  \end{center}
\end{figure}

\begin{itemize}[noitemsep]
\item[$\square$] $-1$
\item[$\square$] $0$
\item[$\square$] $1$
\item[$\square$] $2$
\item[$\square$] $3$
\end{itemize}

\paragraph{L3:} What is $\displaystyle \lim_{x \to 3} \frac{6/x - 2}{3 -4x + x^2}$? (If the limit does not exist, enter DNE.)

\paragraph{L4:} Which of the following functions have a removable discontinuity at $x=2$?

\begin{itemize}
\item[$\square$] $f(x) = \frac{x^2 - x - 2}{x - 2}$
\item[$\square$] $f(x) = \frac{1}{(x-2)^2}$
\item[$\square$] $f(x) = \begin{cases}
      \frac{x^2 - x - 2}{x-2} & x \ne 2\\
      3                       & x = 2
    \end{cases}$
\item[$\square$] $f(x) = \begin{cases}
      x^3 - 1 & x > 2\\
      -x^2    & x \le 2
    \end{cases}$ 
\end{itemize}

\paragraph{L5:} Identify the left-hand limit $\displaystyle \lim_{x \to -1^-} f(x)$ based on the graph of shown below.

\begin{figure}[H]
  \begin{center}
    %\caption{}
    \label{fig:adv-l5}
    \fbox{\includegraphics[scale=0.5]{imagens/adventure_l5.png}}
  \end{center}
\end{figure}

\begin{itemize}[noitemsep]
\item[$\square$] $2$
\item[$\square$] $1$
\item[$\square$] $0$
\item[$\square$] $-1$
\item[$\square$] $-1.5$
\item[$\square$] Does not exist.
\end{itemize}

\paragraph{L6:} Identify the right-handed limit $\displaystyle \lim_{x \to -1^+} \frac{x^2 -1}{|x+1|}$.
(Enter DNE if the limit does not exist.)

\begin{itemize}[noitemsep]
\item If you got 0.65 or above, you have a good handle on limits. Move on to Unit 1.
  You can go back to Unit 0 at any time to fill any gaps in your understanding of limits.
\item If you got between .35--.65 points, we recommend that you start by doing the in-lecture
  problems in Unit 0. You may be able to skip the video tutorials.
\item Otherwise, start with Unit 0!
\end{itemize}

%%%%%%%%%%%%%%%%%%%%%%%%%%%%%%%%%%%%%%%%%%%%%%%%%%%%%%%%%%%%%%%%%%%%%%%%%%%%%%%%%
\subsubsection{Derivatives diagnostic}
\label{gs-adventure-derivative}

You will not see which problems you get correct and incorrect. Sorry if this is frustrating. We will be using these problems to assess whether or not our content teaches you about derivatives.

\paragraph{C1:} What is $\displaystyle \lim_{h \to 0} \frac{\cos{(\pi/6 + h)}-\cos{(\pi/6)}}{h}$? 
(Enter the answer as a decimal. If the limit does not exist, enter DNE.)

\paragraph{C2:} At which of the five points on the graph are $\frac{dy}{dx}$ and $\frac{d^2y}{dx^2}$ both negative?

\begin{figure}[H]
  \begin{center}
    %\caption{}
    \label{fig:adv-c2}
    \fbox{\includegraphics[scale=0.5]{imagens/adventure_c2.png}}
  \end{center}
\end{figure}

\paragraph{C3:} What is the average rate of change of the function $f(x) = x^4 - 5x$ between $x=0$ and $x=3$?

\paragraph{C4:} The position of a particle moving along a line is $p(t) = 2t^3 -24 t^2 +90t + 7$ for $t \ge 0$.
For what values of $t$ is the speed of the particle increasing?

\begin{itemize}[noitemsep]
\item[$\square$] $3 < t < 4$ only
\item[$\square$] $t > 4$ only
\item[$\square$] $t > 5$ only
\item[$\square$] $0 < t < 3$ and $t > 5$
\item[$\square$] $3 < t < 4$ and $t > 5$
\end{itemize}

\paragraph{C5:} Evaluate the limit $\lim_{x \to \infty} \frac{\ln{x}}{x^2}$:

\begin{itemize}[noitemsep]
\item[$\square$] $0$
\item[$\square$] $1$
\item[$\square$] $-1$
\item[$\square$] $\infty$
\item[$\square$] $-\infty$
\end{itemize}

\paragraph{C6:} If $f$ is differentiable at $x=a$, which of the following must be true? Choose all of the following that must be true.

\begin{itemize}[noitemsep]
\item[$\square$] $f$ is continuous at $x=a$.
\item[$\square$] $\lim_{x \to a} f(x)$ exists.
\item[$\square$] $\lim_{x \to a} \frac{f(x) - f(a)}{x-a}$ exists.
\item[$\square$] $f'(a)$ is defined.
\item[$\square$] $f''(a)$ is defined.
\end{itemize}

\paragraph{C7:} Let $f(x)=x^3 + 5x^2 -7x -1$. What is $f'(1)$?

\paragraph{C8:} Let $g(x) = x^2e^x$. What is $g'(1)$?

\paragraph{C9:} Suppose that $f(x) = g(5x)$ for all $x$, and that both functions
are differentiable. Which of the following is necessarily true?

\begin{itemize}[noitemsep]
\item[$\square$] $f'(1) = g'(1)$
\item[$\square$] $f'(5) = g'(1)$
\item[$\square$] $f'(1) = g'(5)$
\item[$\square$] $5f'(1) = g'(1)$
\item[$\square$] $5f'(1) = g'(5)$
\item[$\square$] $f'(1) = 5g'(1)$
\item[$\square$] $f'(1) = 5g'(5)$
\item[$\square$] None of the above.
\end{itemize}

\paragraph{C10:} Let $\displaystyle f(x) = \frac{\ln{(5t+1)}}{\sqrt{t+1}}$. What is $f'(0)$?

If you got 0.8 or above, you have a good handle on the derivative. There is likely no material
in Unit 0 or Unit 1 that you are not familiar with. You may choose to the the
in-lecture exercises, but may wish to skip the video tutorials. If you got less than 0.8, that is to be expected!
We assume you are here to learn about differentiation after all.


%%%%%%%%%%%%%%%%%%%%%%%%%%%%%%%%%%%%%%%%%%%%%%%%%%%%%%%%%%%%%%%%%%%%%%%%%%%%%%%%%
%%%%%%%%%%%%%%%%%%%%%%%%%%%%%%%%%%%%%%%%%%%%%%%%%%%%%%%%%%%%%%%%%%%%%%%%%%%%%%%%%
\subsection{Syllabus and schedule}
\label{gs-syllabus}

%%%%%%%%%%%%%%%%%%%%%%%%%%%%%%%%%%%%%%%%%%%%%%%%%%%%%%%%%%%%%%%%%%%%%%%%%%%%%%%%%
\subsubsection{Syllabus and schedule}
\label{gs-syllabus-syllabus}

\begin{figure}[H]
  \begin{center}
    %\caption{}
    \label{fig:syllabus1}
    \fbox{\includegraphics[scale=0.35]{imagens/syllabus1.png}}
  \end{center}
\end{figure}

\begin{figure}[H]
  \begin{center}
    %\caption{}
    \label{fig:syllabus2}
    \fbox{\includegraphics[scale=0.35]{imagens/syllabus2.png}}
  \end{center}
\end{figure}

\begin{figure}[H]
  \begin{center}
    %\caption{}
    \label{fig:syllabus3}
    \fbox{\includegraphics[scale=0.35]{imagens/syllabus3.png}}
  \end{center}
\end{figure}

\begin{figure}[H]
  \begin{center}
    %\caption{}
    \label{fig:syllabus4}
    \fbox{\includegraphics[scale=0.32]{imagens/syllabus4.png}}
  \end{center}
\end{figure}

\begin{figure}[H]
  \begin{center}
    %\caption{}
    \label{fig:syllabus5}
    \fbox{\includegraphics[scale=0.30]{imagens/syllabus5.png}}
  \end{center}
\end{figure}

\begin{figure}[H]
  \begin{center}
    %\caption{}
    \label{fig:syllabus6}
    \fbox{\includegraphics[scale=0.30]{imagens/syllabus6.png}}
  \end{center}
\end{figure}

\begin{figure}[H]
  \begin{center}
    %\caption{}
    \label{fig:syllabus7}
    \fbox{\includegraphics[scale=0.35]{imagens/syllabus7.png}}
  \end{center}
\end{figure}


%%%%%%%%%%%%%%%%%%%%%%%%%%%%%%%%%%%%%%%%%%%%%%%%%%%%%%%%%%%%%%%%%%%%%%%%%%%%%%%%%
%%%%%%%%%%%%%%%%%%%%%%%%%%%%%%%%%%%%%%%%%%%%%%%%%%%%%%%%%%%%%%%%%%%%%%%%%%%%%%%%%
\subsection{Entrance survey}
\label{gs-survey}

%%%%%%%%%%%%%%%%%%%%%%%%%%%%%%%%%%%%%%%%%%%%%%%%%%%%%%%%%%%%%%%%%%%%%%%%%%%%%%%%%
\subsubsection{Entrance survey}
\label{gs-survey-survey}

Welcome to this online course from MITx.

For us to offer the best course experience possible, we'd like to ask you to answer a few questions about yourself.

Whether you are just browsing or you are determined to complete the entire course, the more we know about you, the better we can serve all students in this course. As one of the first students in this new, free offering, your responses will be especially important to us. 

There are no right or wrong answers or responses, and your honest feedback is very important to us.  After reading the consent document below, you may click the right arrow below to proceed. 

\textbf{General Information About Survey Research in MITx. Please Read then Click Below to Continue.}

\textbf{Participation is voluntary}.
All survey responses are voluntary, students can skip any question at any time, and any responses have no effect on student assessments or participation. 

\textbf{What is the purpose of this research?}
We are interested in learning more about our participants’ backgrounds, interests, and motivations, and encouraging engagement with the course, so we can do the best possible job designing, evaluating and refining this course. With this research we will understand how to best encourage engagement with online education . 

\textbf{How long will I take part in this research?}
Your participation will be the duration of the course.

\textbf{What can I expect if I take part in this research?}
As a participant, you will be provided questions about yourself and other short prompts, which we will use to understand your participation in the course.  
 
\textbf{What are the risks and possible discomforts?}
If you choose to participate, we anticipate minimal risks and only the minor discomfort that might accompany online surveys. 

\textbf{Are there any benefits from being in this research study? }
We cannot promise any benefits to you or others from taking part in this research. However, possible benefits include your being more engaged with the course and better serving future students who participate in online courses. 

\textbf{If I take part in this research, how will my privacy be protected? What happens to the information you collect?}
Your instructor will not be able to identify your personal responses during the course and researchers will not attempt to identify individuals.  Your data will not be made identifiable to anyone other than researchers and course staff, and it will be aggregated for analysis and publication purposes. 

\textbf{If I have any questions, concerns or complaints about this research study, who can I talk to?}
The researcher for this study is Justin Reich who can be reached at 617-715-2962, 600 Technology Square, NE49-2028, Cambridge, MA, 02139, jreich@mit.edu for any of the following:
\begin{itemize}
\item If you have questions, concerns, or complaints,
\item If you would like to talk to the research team,
\item If you think the research has harmed you, or
\item If you wish to withdraw from the study.
\end{itemize}

This research has been reviewed by the Committee on the Use of Human Subjects in Research at Harvard University.  They can be reached at 617-496-2847, 1414 Massachusetts Avenue, Second Floor, Cambridge, MA 02138, or cuhs@fas.harvard.edu for any of the following:
\begin{itemize}
\item If your questions, concerns, or complaints are not being answered by the research team,
\item If you cannot reach the research team,
\item If you want to talk to someone besides the research team, or
\item If you have questions about your rights as a research participant.
\end{itemize}

Please print or save a copy of this form for your records.
\textbf{If you agree to participate, please click "Next" to enter the survey.}

\begin{figure}[H]
  \begin{center}
    %\caption{}
    \label{fig:survey01}
    \fbox{\includegraphics[scale=0.5]{imagens/survey01.png}}
  \end{center}
\end{figure}

\begin{figure}[H]
  \begin{center}
    %\caption{}
    \label{fig:survey02}
    \fbox{\includegraphics[scale=0.5]{imagens/survey02.png}}
  \end{center}
\end{figure}

\begin{figure}[H]
  \begin{center}
    %\caption{}
    \label{fig:survey03}
    \fbox{\includegraphics[scale=0.5]{imagens/survey03.png}}
  \end{center}
\end{figure}

\begin{figure}[H]
  \begin{center}
    %\caption{}
    \label{fig:survey04}
    \fbox{\includegraphics[scale=0.5]{imagens/survey04.png}}
  \end{center}
\end{figure}

\begin{figure}[H]
  \begin{center}
    %\caption{}
    \label{fig:survey05}
    \fbox{\includegraphics[scale=0.5]{imagens/survey05.png}}
  \end{center}
\end{figure}

\begin{figure}[H]
  \begin{center}
    %\caption{}
    \label{fig:survey06}
    \fbox{\includegraphics[scale=0.5]{imagens/survey06.png}}
  \end{center}
\end{figure}

\begin{figure}[H]
  \begin{center}
    %\caption{}
    \label{fig:survey07}
    \fbox{\includegraphics[scale=0.5]{imagens/survey07.png}}
  \end{center}
\end{figure}

\begin{figure}[H]
  \begin{center}
    %\caption{}
    \label{fig:survey08}
    \fbox{\includegraphics[scale=0.5]{imagens/survey08.png}}
  \end{center}
\end{figure}

\begin{figure}[H]
  \begin{center}
    %\caption{}
    \label{fig:survey09}
    \fbox{\includegraphics[scale=0.5]{imagens/survey09.png}}
  \end{center}
\end{figure}

\begin{figure}[H]
  \begin{center}
    %\caption{}
    \label{fig:survey10}
    \fbox{\includegraphics[scale=0.5]{imagens/survey10.png}}
  \end{center}
\end{figure}

\begin{figure}[H]
  \begin{center}
    %\caption{}
    \label{fig:survey11}
    \fbox{\includegraphics[scale=0.5]{imagens/survey11.png}}
  \end{center}
\end{figure}

\begin{figure}[H]
  \begin{center}
    %\caption{}
    \label{fig:survey12}
    \fbox{\includegraphics[scale=0.5]{imagens/survey12.png}}
  \end{center}
\end{figure}

\begin{figure}[H]
  \begin{center}
    %\caption{}
    \label{fig:survey13}
    \fbox{\includegraphics[scale=0.5]{imagens/survey13.png}}
  \end{center}
\end{figure}

\begin{figure}[H]
  \begin{center}
    %\caption{}
    \label{fig:survey14}
    \fbox{\includegraphics[scale=0.5]{imagens/survey14.png}}
  \end{center}
\end{figure}

\begin{figure}[H]
  \begin{center}
    %\caption{}
    \label{fig:survey15}
    \fbox{\includegraphics[scale=0.5]{imagens/survey15.png}}
  \end{center}
\end{figure}

\begin{figure}[H]
  \begin{center}
    %\caption{}
    \label{fig:survey16}
    \fbox{\includegraphics[scale=0.5]{imagens/survey16.png}}
  \end{center}
\end{figure}

\begin{figure}[H]
  \begin{center}
    %\caption{}
    \label{fig:survey17}
    \fbox{\includegraphics[scale=0.5]{imagens/survey17.png}}
  \end{center}
\end{figure}

\begin{figure}[H]
  \begin{center}
    %\caption{}
    \label{fig:survey18}
    \fbox{\includegraphics[scale=0.5]{imagens/survey18.png}}
  \end{center}
\end{figure}

\begin{figure}[H]
  \begin{center}
    %\caption{}
    \label{fig:survey19}
    \fbox{\includegraphics[scale=0.5]{imagens/survey19.png}}
  \end{center}
\end{figure}

\begin{figure}[H]
  \begin{center}
    %\caption{}
    \label{fig:survey20}
    \fbox{\includegraphics[scale=0.5]{imagens/survey20.png}}
  \end{center}
\end{figure}

\begin{figure}[H]
  \begin{center}
    %\caption{}
    \label{fig:survey21}
    \fbox{\includegraphics[scale=0.5]{imagens/survey21.png}}
  \end{center}
\end{figure}

\begin{figure}[H]
  \begin{center}
    %\caption{}
    \label{fig:survey22}
    \fbox{\includegraphics[scale=0.5]{imagens/survey22.png}}
  \end{center}
\end{figure}

\begin{figure}[H]
  \begin{center}
    %\caption{}
    \label{fig:survey23}
    \fbox{\includegraphics[scale=0.5]{imagens/survey23.png}}
  \end{center}
\end{figure}

\begin{figure}[H]
  \begin{center}
    %\caption{}
    \label{fig:survey24}
    \fbox{\includegraphics[scale=0.5]{imagens/survey24.png}}
  \end{center}
\end{figure}

\begin{figure}[H]
  \begin{center}
    %\caption{}
    \label{fig:survey25}
    \fbox{\includegraphics[scale=0.5]{imagens/survey25.png}}
  \end{center}
\end{figure}

\begin{figure}[H]
  \begin{center}
    %\caption{}
    \label{fig:survey26}
    \fbox{\includegraphics[scale=0.5]{imagens/survey26.png}}
  \end{center}
\end{figure}


%%%%%%%%%%%%%%%%%%%%%%%%%%%%%%%%%%%%%%%%%%%%%%%%%%%%%%%%%%%%%%%%%%%%%%%%%%%%%%%%%
%%%%%%%%%%%%%%%%%%%%%%%%%%%%%%%%%%%%%%%%%%%%%%%%%%%%%%%%%%%%%%%%%%%%%%%%%%%%%%%%%
%%%%%%%%%%%%%%%%%%%%%%%%%%%%%%%%%%%%%%%%%%%%%%%%%%%%%%%%%%%%%%%%%%%%%%%%%%%%%%%%%
\section{Unit 0: limits}
\label{u0}

%%%%%%%%%%%%%%%%%%%%%%%%%%%%%%%%%%%%%%%%%%%%%%%%%%%%%%%%%%%%%%%%%%%%%%%%%%%%%%%%%
%%%%%%%%%%%%%%%%%%%%%%%%%%%%%%%%%%%%%%%%%%%%%%%%%%%%%%%%%%%%%%%%%%%%%%%%%%%%%%%%%
\subsection{Introduction to limits}
\label{u0-intro}

%%%%%%%%%%%%%%%%%%%%%%%%%%%%%%%%%%%%%%%%%%%%%%%%%%%%%%%%%%%%%%%%%%%%%%%%%%%%%%%%%
\subsubsection{Motivation}
\label{u0-intro-motiv}

Video: \href{https://www.youtube.com/watch?v=nh5O6-2Evk8}{Introduction to Limits}

Calculus has two main concepts --- the derivative
and the integral.
But in order to understand either of them,
you first have to understand limits.

So let's talk limits.
We'll start with a curve.
Fix a point A on the curve.
Choose a second point, B, which we're going to move.
And draw a line through A and B. Let's look
at what happens when B moves closer and closer to the point
A.

This is an example of a limit.
In the limit, the line becomes tangent to the curve
at the point A. The slope of this line
is the derivative at the point A. Now let's see how limits
are related to integrals.

Integrals are used to measure areas of curvy regions
like this.
Measuring areas of curvy regions seems hard,
but measuring areas of rectangles
is easy, so we'll try to fill our region with rectangles.

Each rectangle has a certain width.
As we make the width smaller, the total area
of the rectangles gets closer and closer
to the area of the curvy region.

The integral is the limit of the total area of the rectangles
as the width tends to zero.

So that's why we start with limits.
They're the foundation for everything else in calculus.
At the beginning, limits may seem abstract,
but very quickly you'll get used to them.

%%%%%%%%%%%%%%%%%%%%%%%%%%%%%%%%%%%%%%%%%%%%%%%%%%%%%%%%%%%%%%%%%%%%%%%%%%%%%%%%%
\subsubsection{Introduction to limits}
\label{u0-intro-intro}

\paragraph{Objectives} At the end of this sequence, and after some practice, you should be able to:
\begin{itemize}[noitemsep]
\item Use a calculator to determine right and left hand limits.
\item Identify right and left hand limits based on graphs.
\item Determine if a limit exists based on values of right and left hand limits.
\item Understand that the limit does not depend on the value of a function at the point of interest.
\end{itemize}

Contents: 14 pages, 6 videos (24 minutes 1x speed), 17 questions.


%%%%%%%%%%%%%%%%%%%%%%%%%%%%%%%%%%%%%%%%%%%%%%%%%%%%%%%%%%%%%%%%%%%%%%%%%%%%%%%%%
\subsubsection{Moving closer and closer}
\label{u0-intro-moving}

Video: \href{https://www.youtube.com/watch?v=bANtYKLugsU}{Moving closer and closer}

Welcome.

Calculus is all about functions. You probably know that a function f takes an input x
and gives an output f of x. But in calculus, we're not concerned with just one input
and finding the output for that one input. We want to consider a whole range of inputs.
So we would want to know what happens when the input "moves" or "varies."
For instance, we could ask what happens as the input moves really close.
Closer and closer to some point. Let's say 1.

And to be even more specific, let's say that x is moving towards 1 from the left.
So if this is a number line, and we've got the point 1 right there, then x
could start here, and just move closer and closer and closer towards 1, from the left.
We'll use this arrow notation to denote that x is getting really, really close to 1.
But a warning, this does not mean that x will ever actually equal 1.
We're only concerned with values of x that are near one.

OK. Now that that's said, as x moves,
we know that the output f of x is also going to move.
And so the question that we can ask
is as x moves closer and closer to 1 from the left,
does f of x move closer and closer
to some value of its own?

Let's be concrete here.
And pick a particular function f.
I'm going to choose f of x to be the square root of 3 minus 5 x
plus x squared, plus x cubed, all over x minus 1.
Kind of a complicated function, but you'll
have to trust me that this is a good example.
And what we can do in order to see what's
happening to f as x approaches 1 from the left
is just select certain values of x
that are getting closer and closer to 1 from the left.
So over here on the number line, we
could start with x equals zero.
And then they get closer, we could try x equals 0.5.
Or even closer, maybe 0.9.
Even 0.99.
These sorts of values.
And we want to know, what's happening to the output?
So we can just plug these values into the function,
and see whether the output gets closer and closer to anything.

Now there are technically infinitely many values
of x that we could have chosen here.
But let's just start with these four.
Remember though that one value of x
that we will definitely not consider
is x equal to 1 itself.
In fact, this function isn't even defined at x equals 1.
We'd have a zero denominator.
It is, however, defined when x is approaching one,
and those are the values we're considering.

OK.
Well let's make a table with our chosen
inputs and the associated outputs,
and let's just calculate those outputs.
So when we plug in zero we'll get a square root of 3
on top divided by minus 1.
So minus square root of 3, which is roughly minus 1.73.
Next up is x equals 0.5.
I'm going to have to bust out the calculator here.
So we've got 3 minus 5 times 0.5 plus 0.5 squared
plus 0.5 cubed, and then we need the square root,
and then we need to divide by 0.5 minus 1.
So 0.5 negative.
So we get minus 1.87, roughly.
So back to our table.
We've got f of x moving from minus 1.73 to minus 1.87.
Well that's not really enough data
to tell if f is getting closer and closer to anything
in particular.
So let's take our next two values of x and plug those in.
I'm going to fast forward through the calculations.
You ready?
x equals 0.9.
All right?
That's approximately minus 1.97, and finally 0.99,
and we've got minus 1.997.
So as we go down this table, f of x
is getting really, really close to what looks like minus 2.
So we can say that as x approaches 1 from the left,
f of x approaches minus 2.
Now f of x might never actually equal to minus 2,
just as x never actually equals one,
but it gets really, really close.
And if it gets arbitrarily close,
meaning as close as we could possibly want,
then that's really all we'll care about.
What I would like you to do now is
to do this same exercise, except this time have x approach 1
from the right.
You might be surprised at what you find.
We'll talk afterwards.

Determine what happens to $\displaystyle f(x) = \frac{\sqrt{3-5x+x^2+x^3}}{x-1}$ as $x$ approaches $1$ from the right.
Take values of $x$ that are greater than 1, but getting closer and closer to 1. For instance, you could try
$x=1.1, 1.01, 1.001, 1.0001$, etc. What happens to $f(x)$ as $x$ approaches 1 from the right?

\begin{itemize}[noitemsep]
\item[$\square$] $f(x)$ gets closer and closer to a particular number
\item[$\square$] $f(x)$ gets bigger and bigger in the positive direction without bound
\item[$\square$] $f(x)$ gets bigger and bigger in the negative direction without bound
\item[$\square$] None of the above
\end{itemize}

What value does $f(x)$ get closer to? Enter the number below;
if there is no such value, enter capital (for "does not exist")

%%%%%%%%%%%%%%%%%%%%%%%%%%%%%%%%%%%%%%%%%%%%%%%%%%%%%%%%%%%%%%%%%%%%%%%%%%%%%%%%%
\subsubsection{One-sided limits}
\label{u0-intro-one-sided}

Video: \href{https://www.youtube.com/watch?v=fAAzAVHVKQk}{One-sided limits}

%%%%%%%%%%%%%%%%%%%%%%%%%%%%%%%%%%%%%%%%%%%%%%%%%%%%%%%%%%%%%%%%%%%%%%%%%%%%%%%%%
\subsubsection{Definitions of right-hand and left-hand limits}
\label{u0-intro-right-left}

%%%%%%%%%%%%%%%%%%%%%%%%%%%%%%%%%%%%%%%%%%%%%%%%%%%%%%%%%%%%%%%%%%%%%%%%%%%%%%%%%
\subsubsection{A few more limits}
\label{u0-intro-more}

%%%%%%%%%%%%%%%%%%%%%%%%%%%%%%%%%%%%%%%%%%%%%%%%%%%%%%%%%%%%%%%%%%%%%%%%%%%%%%%%%
\subsubsection{Possible limits behaviors}
\label{u0-intro-behaviors}

%%%%%%%%%%%%%%%%%%%%%%%%%%%%%%%%%%%%%%%%%%%%%%%%%%%%%%%%%%%%%%%%%%%%%%%%%%%%%%%%%
\subsubsection{Quick limit questions}
\label{u0-intro-questions}

%%%%%%%%%%%%%%%%%%%%%%%%%%%%%%%%%%%%%%%%%%%%%%%%%%%%%%%%%%%%%%%%%%%%%%%%%%%%%%%%%
\subsubsection{The overall limit}
\label{u0-intro-overall}

%%%%%%%%%%%%%%%%%%%%%%%%%%%%%%%%%%%%%%%%%%%%%%%%%%%%%%%%%%%%%%%%%%%%%%%%%%%%%%%%%
\subsubsection{Limit definition}
\label{u0-intro-definition}

%%%%%%%%%%%%%%%%%%%%%%%%%%%%%%%%%%%%%%%%%%%%%%%%%%%%%%%%%%%%%%%%%%%%%%%%%%%%%%%%%
\subsubsection{Limits from graphs}
\label{u0-intro-graphs}

%%%%%%%%%%%%%%%%%%%%%%%%%%%%%%%%%%%%%%%%%%%%%%%%%%%%%%%%%%%%%%%%%%%%%%%%%%%%%%%%%
\subsubsection{Review problems}
\label{u0-intro-review}

%%%%%%%%%%%%%%%%%%%%%%%%%%%%%%%%%%%%%%%%%%%%%%%%%%%%%%%%%%%%%%%%%%%%%%%%%%%%%%%%%
\subsubsection{Limit laws}
\label{u0-intro-laws}

%%%%%%%%%%%%%%%%%%%%%%%%%%%%%%%%%%%%%%%%%%%%%%%%%%%%%%%%%%%%%%%%%%%%%%%%%%%%%%%%%
\subsubsection{Limit laws (2)}
\label{u0-intro-laws2}

%%%%%%%%%%%%%%%%%%%%%%%%%%%%%%%%%%%%%%%%%%%%%%%%%%%%%%%%%%%%%%%%%%%%%%%%%%%%%%%%%
\subsubsection{Summary}
\label{u0-intro-summary}


%%%%%%%%%%%%%%%%%%%%%%%%%%%%%%%%%%%%%%%%%%%%%%%%%%%%%%%%%%%%%%%%%%%%%%%%%%%%%%%%%
%%%%%%%%%%%%%%%%%%%%%%%%%%%%%%%%%%%%%%%%%%%%%%%%%%%%%%%%%%%%%%%%%%%%%%%%%%%%%%%%%
\subsection{Continuity}
\label{u0-cont}

%%%%%%%%%%%%%%%%%%%%%%%%%%%%%%%%%%%%%%%%%%%%%%%%%%%%%%%%%%%%%%%%%%%%%%%%%%%%%%%%%
\subsubsection{Motivation}
\label{u0-cont-motivation}

%%%%%%%%%%%%%%%%%%%%%%%%%%%%%%%%%%%%%%%%%%%%%%%%%%%%%%%%%%%%%%%%%%%%%%%%%%%%%%%%%
\subsubsection{How do we compute limits?}
\label{u0-cont-how-compute}

%%%%%%%%%%%%%%%%%%%%%%%%%%%%%%%%%%%%%%%%%%%%%%%%%%%%%%%%%%%%%%%%%%%%%%%%%%%%%%%%%
\subsubsection{Continuity}
\label{u0-cont-continuity}

%%%%%%%%%%%%%%%%%%%%%%%%%%%%%%%%%%%%%%%%%%%%%%%%%%%%%%%%%%%%%%%%%%%%%%%%%%%%%%%%%
\subsubsection{Continuity questions}
\label{u0-cont-continuity-questions}

%%%%%%%%%%%%%%%%%%%%%%%%%%%%%%%%%%%%%%%%%%%%%%%%%%%%%%%%%%%%%%%%%%%%%%%%%%%%%%%%%
\subsubsection{More continuity questions}
\label{u0-cont-more-continuity-questions}

%%%%%%%%%%%%%%%%%%%%%%%%%%%%%%%%%%%%%%%%%%%%%%%%%%%%%%%%%%%%%%%%%%%%%%%%%%%%%%%%%
\subsubsection{Overall continuity}
\label{u0-cont-overall}

%%%%%%%%%%%%%%%%%%%%%%%%%%%%%%%%%%%%%%%%%%%%%%%%%%%%%%%%%%%%%%%%%%%%%%%%%%%%%%%%%
\subsubsection{Continuity continued}
\label{u0-cont-continuity-continued}

%%%%%%%%%%%%%%%%%%%%%%%%%%%%%%%%%%%%%%%%%%%%%%%%%%%%%%%%%%%%%%%%%%%%%%%%%%%%%%%%%
\subsubsection{Limit laws and continuity}
\label{u0-cont-laws}

%%%%%%%%%%%%%%%%%%%%%%%%%%%%%%%%%%%%%%%%%%%%%%%%%%%%%%%%%%%%%%%%%%%%%%%%%%%%%%%%%
\subsubsection{Review of continuity}
\label{u0-cont-review}

%%%%%%%%%%%%%%%%%%%%%%%%%%%%%%%%%%%%%%%%%%%%%%%%%%%%%%%%%%%%%%%%%%%%%%%%%%%%%%%%%
\subsubsection{Catalog of continuous functions}
\label{u0-cont-catalog}

%%%%%%%%%%%%%%%%%%%%%%%%%%%%%%%%%%%%%%%%%%%%%%%%%%%%%%%%%%%%%%%%%%%%%%%%%%%%%%%%%
\subsubsection{IVT intro}
\label{u0-cont-ivt}

%%%%%%%%%%%%%%%%%%%%%%%%%%%%%%%%%%%%%%%%%%%%%%%%%%%%%%%%%%%%%%%%%%%%%%%%%%%%%%%%%
\subsubsection{Intermediate Value Theorem}
\label{u0-cont-intermediate-value-theorem}

%%%%%%%%%%%%%%%%%%%%%%%%%%%%%%%%%%%%%%%%%%%%%%%%%%%%%%%%%%%%%%%%%%%%%%%%%%%%%%%%%
\subsubsection{Basics}
\label{u0-cont-basics}

%%%%%%%%%%%%%%%%%%%%%%%%%%%%%%%%%%%%%%%%%%%%%%%%%%%%%%%%%%%%%%%%%%%%%%%%%%%%%%%%%
\subsubsection{Roots}
\label{u0-cont-roots}

%%%%%%%%%%%%%%%%%%%%%%%%%%%%%%%%%%%%%%%%%%%%%%%%%%%%%%%%%%%%%%%%%%%%%%%%%%%%%%%%%
\subsubsection{Summary}
\label{u0-cont-summary}


%%%%%%%%%%%%%%%%%%%%%%%%%%%%%%%%%%%%%%%%%%%%%%%%%%%%%%%%%%%%%%%%%%%%%%%%%%%%%%%%%
%%%%%%%%%%%%%%%%%%%%%%%%%%%%%%%%%%%%%%%%%%%%%%%%%%%%%%%%%%%%%%%%%%%%%%%%%%%%%%%%%
\subsection{Limits of quotients}
\label{u0-lim-quo}

%%%%%%%%%%%%%%%%%%%%%%%%%%%%%%%%%%%%%%%%%%%%%%%%%%%%%%%%%%%%%%%%%%%%%%%%%%%%%%%%%
\subsubsection{Limits of quotients}
\label{u0-lim-quo-lim}

%%%%%%%%%%%%%%%%%%%%%%%%%%%%%%%%%%%%%%%%%%%%%%%%%%%%%%%%%%%%%%%%%%%%%%%%%%%%%%%%%
\subsubsection{How do we compute limits of quotients?}
\label{u0-lim-quo-how-compute}

%%%%%%%%%%%%%%%%%%%%%%%%%%%%%%%%%%%%%%%%%%%%%%%%%%%%%%%%%%%%%%%%%%%%%%%%%%%%%%%%%
\subsubsection{Limits and division}
\label{u0-lim-quo-lim-div}

%%%%%%%%%%%%%%%%%%%%%%%%%%%%%%%%%%%%%%%%%%%%%%%%%%%%%%%%%%%%%%%%%%%%%%%%%%%%%%%%%
\subsubsection{Quotients of small numbers}
\label{u0-lim-quo-small}

%%%%%%%%%%%%%%%%%%%%%%%%%%%%%%%%%%%%%%%%%%%%%%%%%%%%%%%%%%%%%%%%%%%%%%%%%%%%%%%%%
\subsubsection{Small divided by small}
\label{u0-lim-quo-small-div-small}

%%%%%%%%%%%%%%%%%%%%%%%%%%%%%%%%%%%%%%%%%%%%%%%%%%%%%%%%%%%%%%%%%%%%%%%%%%%%%%%%%
\subsubsection{Limit law for division}
\label{u0-lim-quo-limit-law-div}

%%%%%%%%%%%%%%%%%%%%%%%%%%%%%%%%%%%%%%%%%%%%%%%%%%%%%%%%%%%%%%%%%%%%%%%%%%%%%%%%%
\subsubsection{Limit laws}
\label{u0-lim-quo-limit-laws}

%%%%%%%%%%%%%%%%%%%%%%%%%%%%%%%%%%%%%%%%%%%%%%%%%%%%%%%%%%%%%%%%%%%%%%%%%%%%%%%%%
\subsubsection{Using the division limit law}
\label{u0-lim-quo-using-law}

%%%%%%%%%%%%%%%%%%%%%%%%%%%%%%%%%%%%%%%%%%%%%%%%%%%%%%%%%%%%%%%%%%%%%%%%%%%%%%%%%
\subsubsection{Division limit questions}
\label{u0-lim-quo-questions}

%%%%%%%%%%%%%%%%%%%%%%%%%%%%%%%%%%%%%%%%%%%%%%%%%%%%%%%%%%%%%%%%%%%%%%%%%%%%%%%%%
\subsubsection{Review problems}
\label{u0-lim-quo-review-prob}

%%%%%%%%%%%%%%%%%%%%%%%%%%%%%%%%%%%%%%%%%%%%%%%%%%%%%%%%%%%%%%%%%%%%%%%%%%%%%%%%%
\subsubsection{Limits that don't exist}
\label{u0-lim-quo-dont-exist}

%%%%%%%%%%%%%%%%%%%%%%%%%%%%%%%%%%%%%%%%%%%%%%%%%%%%%%%%%%%%%%%%%%%%%%%%%%%%%%%%%
\subsubsection{Another function}
\label{u0-lim-quo-another}

%%%%%%%%%%%%%%%%%%%%%%%%%%%%%%%%%%%%%%%%%%%%%%%%%%%%%%%%%%%%%%%%%%%%%%%%%%%%%%%%%
\subsubsection{Infinite limits 2}
\label{u0-lim-quo-inf2}

%%%%%%%%%%%%%%%%%%%%%%%%%%%%%%%%%%%%%%%%%%%%%%%%%%%%%%%%%%%%%%%%%%%%%%%%%%%%%%%%%
\subsubsection{What is the limit?}
\label{u0-lim-quo-what}

%%%%%%%%%%%%%%%%%%%%%%%%%%%%%%%%%%%%%%%%%%%%%%%%%%%%%%%%%%%%%%%%%%%%%%%%%%%%%%%%%
\subsubsection{Division involving infinite limits}
\label{u0-lim-quo-div-inf}

%%%%%%%%%%%%%%%%%%%%%%%%%%%%%%%%%%%%%%%%%%%%%%%%%%%%%%%%%%%%%%%%%%%%%%%%%%%%%%%%%
\subsubsection{Summary}
\label{u0-lim-quo-summary}


%%%%%%%%%%%%%%%%%%%%%%%%%%%%%%%%%%%%%%%%%%%%%%%%%%%%%%%%%%%%%%%%%%%%%%%%%%%%%%%%%
%%%%%%%%%%%%%%%%%%%%%%%%%%%%%%%%%%%%%%%%%%%%%%%%%%%%%%%%%%%%%%%%%%%%%%%%%%%%%%%%%
\subsection{Homework Unit 0: Part A}
\label{u0-hw-pA}

%%%%%%%%%%%%%%%%%%%%%%%%%%%%%%%%%%%%%%%%%%%%%%%%%%%%%%%%%%%%%%%%%%%%%%%%%%%%%%%%%
\subsubsection{Part A Problems}
\label{u0-hw-pA-problems}

%%%%%%%%%%%%%%%%%%%%%%%%%%%%%%%%%%%%%%%%%%%%%%%%%%%%%%%%%%%%%%%%%%%%%%%%%%%%%%%%%
\subsubsection{Part A Homework}
\label{u0-hw-pA-homework}










%%%%%%%%%%%%%%%%%%%%%%%%%%%%%%%%%%%%%%%%%%%%%%%%%%%%%%%%%%%%%%%%%%%%%%%%%%%%%%%%%
%%%%%%%%%%%%%%%%%%%%%%%%%%%%%%%%%%%%%%%%%%%%%%%%%%%%%%%%%%%%%%%%%%%%%%%%%%%%%%%%%
%%%%%%%%%%%%%%%%%%%%%%%%%%%%%%%%%%%%%%%%%%%%%%%%%%%%%%%%%%%%%%%%%%%%%%%%%%%%%%%%%
%%%%%%%%%%%%%%%%%%%%%%%%%%%%%%%%%%%%%%%%%%%%%%%%%%%%%%%%%%%%%%%%%%%%%%%%%%%%%%%%%
%%%%%%%%%%%%%%%%%%%%%%%%%%%%%% TERMINA O DOCUMENTO %%%%%%%%%%%%%%%%%%%%%%%%%%%%%%
%%%%%%%%%%%%%%%%%%%%%%%%%%%%%%%%%%%%%%%%%%%%%%%%%%%%%%%%%%%%%%%%%%%%%%%%%%%%%%%%%
%%%%%%%%%%%%%%%%%%%%%%%%%%%%%%%%%%%%%%%%%%%%%%%%%%%%%%%%%%%%%%%%%%%%%%%%%%%%%%%%%
%%%%%%%%%%%%%%%%%%%%%%%%%%%%%%%%%%%%%%%%%%%%%%%%%%%%%%%%%%%%%%%%%%%%%%%%%%%%%%%%%
%%%%%%%%%%%%%%%%%%%%%%%%%%%%%%%%%%%%%%%%%%%%%%%%%%%%%%%%%%%%%%%%%%%%%%%%%%%%%%%%%
\end{document}









